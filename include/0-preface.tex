% cleardoublepage and phantomsection force that link jumps ABOVE heading, normally it would jump UNDER heading
\cleardoublepage\phantomsection

\addcontentsline{toc}{chapter}{Preface}
\chapter*{Preface}
\markboth{\MakeUppercase{Preface}}{}

% česky Předmluva ...
% should be arresting

Cryptography emerged long long time ago with the need for confidential communication in the presence of enemies -- Caesar cipher can serve as a well-known example of a primitive cipher. Cryptography also had considerable consequences in the past -- for instance Mary, Queen of Scots, was executed back in \nth{16} century, because she was planning to assasinate Queen Elizabeth I of England and her cipher got broken. However, the most appreciable success of cryptography was definitely the breach of Nazi Germany's Enigma code, which Nazi's fortunately beleived is unbreakable.

Since its revival few decades ago, cryptography became a proper discipline of modern mathematics and now it covers much more than ``hiding secrets''. For instance, there are cryptographic primitives which allow us to construct digital signatures, message authentication codes or even more complicated structures like digital currencies.

In this thesis, we will focus on another interesting problem: let us imagine that we would like to work with confidential data in an untrusted environment. For example we would like to let an untrusted party process our confidential data without learning anything about it, or -- which we will study in this thesis -- we would like to let anybody use a cryptographic key without being able to recover it. Such a property of an implementation of a cipher will be referred to as {\em white-box attack resistency}. However, white-box attack resistancy appears to be quite hard to achieve.

The primary goal of this thesis is to reproduce, improve and analyze a recently introduced attack technique against white-box attack protected implementations.


% In this thesis, we present one such variant \cite{chow2002aes} of a standard cipher \cite{fips2001aes} together with its C++ implementation \cite{klinec2013implementation}. Then we introduce a novel powerful attack \cite{bos2015differential}, which we further improve and make even more powerful. Finally we give a very likely explanation why the attack is actually possible.

\section*{Work overview}
\markright{\MakeUppercase{Work overview}}
	\begin{description}
		\item[Chapter \ref{chap:intro}: \nameref{chap:intro}.] ~ \\
			First of all, we give an introduction to cryptography.
		\item[Chapter \ref{chap:attack}: \nameref{chap:attack}.] ~ \\
			We present the attack technique.
		\item[Chapter \ref{chap:results}: \nameref{chap:results}.] ~ \\
			First we reproduce results of Bos et al., then we suggest and run an extended variant of the attack.
		\item[Chapter \ref{chap:analysis}: \nameref{chap:analysis}.] ~ \\
			In the fourth chapter, we analyze.
		\item[Chapter \ref{chap:future}: \nameref{chap:future}.] ~ \\
			Finally, we summarize possible topics of further research.
	\end{description}
