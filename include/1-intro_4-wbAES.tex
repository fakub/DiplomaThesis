\section{White-Box AES}
\label{sec:wbaes}

In this section, we describe the White-Box AES (WBAES) as introduced by Chow et al.\ \cite{chow2002aes} and revisited by Muir \cite{muir2013tutorial} (which we highly recommend). Note that it has already been broken by an interesting algebraic attack by Billet et al.\ \cite{billet2004cryptanalysis}, which will be mentioned further in Section~\ref{sec:known}.

The core idea of WBAES is based on the lookup table implementation as outlined in Section~\ref{sec:aeslookup}, these tables are further improved in three ways:
\begin{enumerate}
	\item the implementation is turned into a {\em fully} table representation, i.e., the key is incorporated into tables as well,
	\item tables are wrapped around with appropriately self-vanishing random bijections, \label{item:wrap}
	\item the boundary is extended into the containing application. \label{item:boundary}
\end{enumerate}
Note that Chow et al.\ introduced also the White-Box DES \cite{chow2002des}, but it has been broken by Jacob et al.\ \cite{jacob2002attacking} much earlier than WBAES, because it was lacking feature~\ref{item:boundary} from the previous list.

\begin{notation}
\label{notation:table}
	Since WBAES will use lookup tables heavily, let us introduce a simplifying notation: a table will be told to have dimensions $m\times n$ if it has $m$-bit input values and $n$-bit output values. Note that the real size (i.e., memory consumption) of such table is $2^m\cdot n$ bits, since we need $2^m$ entries of size $n$ to treat all possible $m$-bit input values.
\end{notation}


% ==============================================================================
% ===   E N C O D I N G S                                                    ===
% ==============================================================================

\subsection{Encodings}
\label{sec:encod}

Let us first introduce some notion for the feature~\ref{item:wrap} from the previous list as defined by Chow et al.

\begin{defn}[Encoding]
	Let $X$ be a transformation from $m$ bits to $n$ bits. Choose an $m$-bit bijection $F$ and an $n$-bit bijection $G$. Call $X' = G \circ X \circ F^{-1}$ an {\em encoded} version of $X$. $F$ is an {\em input encoding} and $G$ is an {\em output encoding}.
\end{defn}

\begin{note}
	Given a fixed key, AES itself is a $128$-bit bijection, which can obviously hardly be implemented as a lookup table, because its size would be $2^{128}\times 128$ bits. On the other hand, this would be black-box equivalent -- we only know the input/output behavior.
	
	Hence we will rather only consider a special kind of bijections -- we will use series of smaller bijections and concatenate their outputs, a definition follows.
\end{note}

\begin{defn}[Concatenated Encoding]
\label{def:concat}
	Let $F_i$ be $n_i$-bit bijections for $i=\atob{1}{k}$ and let $n = n_1 + \ldots + n_k$. The {\em function concatenation} $F_1 \| F_2 \| \ldots \| F_k$ is the $n$-bit bijection $F$ such that $F(b) = F_1(\atob{b_1}{b_{n_1}})\| F_2(\atob{b_{n_1+1}}{b_{n_1+n_2}})\|\ldots\| F_k(\atob{b_{n_1+\ldots+n_{k-1}+1}}{b_n})$ for any $n$-bit word $b=b_1\|b_2\|\ldots\|b_n$, where $\|$ denotes word concatenation (might be omitted). Clearly $F^{-1} = F_1^{-1}\| F_2^{-1}\| \ldots \|F_k^{-1}$.
\end{defn}

The following definition introduces the self-vanishing property of the bijections.

\begin{defn}[Networked Encoding]
\label{def:netw}
	A {\em networked encoding} for computing $Y\circ X$ (i.e., transformation $X$ followed by transformation $Y$) is an encoding of the form
	\[
		Y'\circ X' = (H\circ Y\circ G^{-1})\circ(G\circ X\circ F^{-1}) = H\circ(Y\circ X)\circ F^{-1}
	\]
	where $F$, $G$, $H$ are bijections of appropriate size, $F$ and $H$ are {\em external encodings} (input and output encoding, respectively) and $G$ is an {\em internal encoding}.
\end{defn}

Note that, given only tables for $Y'$ and $X'$, $G$ can be totally forgotten and we are still able to compute $Y\circ X$, provided that we know the input and output encoding. Such a networked encoding can be obviously applied to much longer composition of transformations and we still only need to know the input and output encoding.


% ==============================================================================
% ===   R E O R D E R E D   A E S                                            ===
% ==============================================================================

\subsection{Reordered AES}

In order to begin with fully table representation, we need to reorder AES operations, so that the resulting tables can be easily composed. Let us first have a brief look at AES -- Algorithm~\ref{alg:aes}.

We ``move'' the {\tt for} cycle on Line~\ref{line:aesfor} one row upwards with appropriate shift of indexes in $\ExpandedKey$. Then we switch $\ShiftRows$ and $\SubBytes$ without any side-effect since $\SubBytes$ operates byte-wise. Finally, we switch $\ShiftRows$ and $\AddRoundKey$, while we moreover have to apply $\ShiftRows$ on $\ExpandedKey$ in order to get an equivalent algorithm. The modified version looks as follows:
\begin{alg}
\label{alg:reordaes}
	~
	\begin{algorithmic}[1]
		\Function{Reordered\_AES\_Encryption}{$Plaintext,Key$}
			\State $\ExpandedKey \gets \KeySchedule(Key)$
			\State $State \gets Plaintext$
			\For{$Round = 0 \to 8$}
				\State $\ShiftRows(State)$
				\State $\AddRoundKey(State, \ShiftRows(\ExpandedKey[Round]))$
				\State $\SubBytes(State)$
				\State $\MixColumns(State)$
			\EndFor
			\State $\ShiftRows(State)$
			\State $\AddRoundKey(State, \ShiftRows(\ExpandedKey[9]))$
			\State $\SubBytes(State)$
			\State $\AddRoundKey(State, \ExpandedKey[10])$
			\State\Return $State$
		\EndFunction
	\end{algorithmic}
\end{alg}


% ==============================================================================
% ===   F U L L Y - T A B L E   R E P R E S E N T A T I O N                  ===
% ==============================================================================

\newpage   %_%

\subsection{Fully Table Representation}

As outlined earlier, WBAES construction turns all AES operations into table lookups only. Moreover, we compose certain tables and wrap them all around with internal encodings in the fashion of Definition~\ref{def:netw}. WBAES generator thus inputs an AES key and a random seed, and outputs key-dependent tables which serve as a WBAES instance.

\begin{note}
\label{note:tableinst}
	Some tables of a given type (e.g., $\SubBytes$ table) might be equal across or within rounds, but we will consider every possible instance of each table, since we will give them later distinct encodings.
\end{note}

\subsubsection{T-Boxes}
	
	First, we compose $\AddRoundKey$ with the subsequent $\SubBytes$ step, yielding an operation referred to as $\TBox$, also denoted by $T_{i,j}^r$ for $r$-th round and $(i,j)$-th position in the state array. We get
	\begin{equation*}
		T_{i,j}^r(x) = S(x+k_{sr(i,j)}^r) , \qquad i,j=\atob{0}{3},\;r=\atob{0}{8} ,
	\end{equation*}
	where $k_{sr(i,j)}^r$ stands for the $\ExpandedKey$'s byte in the $r$-th round at the position $(i,j)$ shifted by $\ShiftRows$ (here denoted by $sr(i,j)$). The last round is treated individually as
	\begin{equation}
	\label{eq:lasttbox}
		T_{i,j}^{9}(x) = S(x+k_{sr(i,j)}^9) + k_{i,j}^{10} , \qquad i,j=\atob{0}{3} .
	\end{equation}
	
	Single $\TBox$ can be implemented as an $8\times 8$ lookup table\footnote{Reminder of Notation~\ref{notation:table}.}, and there are $16\cdot 10 = 160$ of them.

\subsubsection{MixColumns}
	
	$\MixColumns$ (here also denoted by $\MC$), acting on a single column, can be turned into four $8\times 32$ lookup tables followed by three XOR operations as outlined in Section~\ref{sec:aeslookup}. Since there are $9$ uses of $\MixColumns$, each acts on $4$ columns, and each such requires $4$ tables, $\MixColumns$ requires $144$ tables altogether.
	
	The difference from the classical implementation is that the subsequent XOR operations must be turned into lookup tables as well. Since the input size of such XOR operation would be $2\cdot 32$ bits, we need to split inputs into smaller $4$-bit segments and treat them separately in the fashion of concatenated encoding (Definition~\ref{def:concat}).
	
	The size of such a single XOR table is then $2\cdot 4\times 4$, and we need to perform $32/4\cdot 3=24$ such lookups per one column and per one $\MixColumns$. It follows that we need altogether $24\cdot 4\cdot 9 = 864$ of such XOR tables (according to Note~\ref{note:tableinst}).

\subsubsection{ShiftRows}
	
	$\ShiftRows$ does not need table representation, since it only moves bytes within the state.

\subsubsection{Table Composition}
	
	Note that $\TBox$ is implemented as an $8\times 8$ table, which is followed by an $8\times 32$ table of $\MixColumns$ in rounds $\atob{0}{8}$. Therefore it is reasonable to compose those appropriately subsequent tables together -- we save both space and time.
	
	It follows that altogether we have $144$ composed $\TBox\circ\MixColumns$ tables, $864$ XOR tables and $16$ final round $\TBox$ tables.


% ==============================================================================
% ===   W B A E S   C O N S T R U C T I O N                                  ===
% ==============================================================================

\subsection{WBAES Construction}
\label{sec:constrwbaes}

So far, we have turned all AES operations into lookup tables only. Next we will incorporate encodings as described in Section~\ref{sec:encod}, these encodings will play the role of a confusion feature. Then introduce a diffusion feature called {\em Mixing Bijection}, and finally compose certain tables together.

\subsubsection{Inserting Encodings}
	
	Note that the $32$-bit outputs of $\MixColumns$ are first split into $4$-bit words, and then two independent $4$-bit words are XORed together, yielding a new $4$-bit word. Hence we are limited to use (only) $4$-bit internal encodings, here denoted by $\Enc$, on both sides of each XOR. This is the reason why {\em input mixing bijections} will be introduced -- they will diffuse two neighboring $4$-bit blocks together.

\subsubsection{Mixing Bijections}
	
	Mixing bijection (denoted by $\MB$) is a random column-wise (i.e., $32$-bit) linear bijection that is to be inserted after $\MC$ and inverted in a separate step (this needs to be done before applying encodings). Since it inputs the output of $\MC$, they can be composed together, yielding $\MB\circ\MC$. Note that table representation of both $\MB\circ\MC$ and $\MB^{-1}$ can be created in very the same manner as previously for $\MC$.
	
	\begin{note}
	\label{note:fullrank}
		Chow et al.\ claim that all aligned $4\times4$ submatrices of $\MB$ must have full rank. This requirement is related to using $4$-bit encodings introduced previously.
	\end{note}

\subsubsection{Input Mixing Bijections}
	
	Input mixing bijection ($\IMB$) is a byte-wise variant of mixing bijection. It provides diffusion after inverting mixing bijection and before entering next $\TBox$. Since it is a linear mapping, it can be composed with the inverted mixing bijection, and since it is a byte-wise mapping, it can be composed with $\TBox$ as well.
	
	The only remaining operation -- $\ShiftRows$ -- will be treated separately later.

\subsubsection{Sketch of Single Round}
	
	Let us demonstrate previous ideas in a single sketch of the flow through one inner round (i.e., not the first round nor the last) showing all operations, table compositions, mixing bijections and encodings. Tables are given their names (e.g., Type II) as introduced by Chow et al., and ``plain'' stands for any intermediate result of the original AES (i.e., without being hidden by another mapping).
	\begin{align}
	\label{eq:wbaesflow}
		\ldots&\rarr\underbracket{\Enc\rarr\IMB^{-1}\xrightarrow{\textnormal{plain}}\TBox\xrightarrow{\textnormal{plain}}\MB\circ\MC\rarr\Enc^{-1}}_{\textnormal{Type II}}\rarr\Bigl(\underbracket{\Enc\rarr\xor\rarr\Enc^{-1}}_{\textnormal{Type IV}}\Bigr)^2\rarr\nonumber\\[.5em]
		&\rarr\underbracket{\Enc\rarr\IMB\circ\MB^{-1}\rarr\Enc^{-1}}_{\textnormal{Type III}}\rarr\Bigl(\underbracket{\Enc\rarr\xor\rarr\Enc^{-1}}_{\textnormal{Type IV}}\Bigr)^2\rarr\ldots
	\end{align}
	Interested reader is referred to Miur's tutorial \cite{muir2013tutorial}, where they give large and well-arranged figures depicting the flow in detail, including I/O bitsizes or depicting incorporation of $\ShiftRows$ operation. Note that I/O bitsizes differ across WBAES mappings, and Equation~\ref{eq:wbaesflow} does not provide that information.
	
\subsubsection{ShiftRows}
	
	$\ShiftRows$ only moves bytes within the state. Indeed, at the time of applying $\ShiftRows$ (i.e., before entering Type II table), only encoding and input mixing bijection are applied, both operate byte-wise.

\subsubsection{External Encodings}
	
	So far, we have only treated internal rounds, i.e., rounds $\atob{1}{8}$ in reordered AES (see Algorithm~\ref{alg:reordaes}). There is one more issue with tables on the very input and on the very output -- these can handle either non-encoded, or appropriately encoded data. Chow et al.\ introduced {\em external encodings} for the latter and gave it a reasoning: as another level of obfuscation, encoded plaintexts and ciphertexts are used instead of raw ones; encoding can then be hidden within a larger containing application or implemented separately.
	
	Let $E_k$ stand for AES encryption using key $k$, then encryption using external encodings can be written as
	\begin{equation}
	\label{eq:extenc}
		E_k' = G \circ E_k \circ F^{-1} ,
	\end{equation}
	where $G$, $F$ stands for input and output external encoding, respectively. These are basically $128$-bit bijections. Clearly, the input plaintext must first be encoded with $F$, then encrypted with encoded variant $E_k'$, and finally decoded with $G^{-1}$, in order to get an equivalent of AES encryption.
	
	Chow et al.\ suggest to use $128$-bit linear mappings as external encodings. These can be further composed with corresponding input mixing bijections of the first and the last round, respectively, preserving linearity.
	
	Such external encodings can be obviously implemented as lookup tables in a manner similar to $\MixColumns$, here using $16$ tables\footnote{A new, larger type of tables is needed.} of size $8\times 128$ followed by appropriately encoded XOR tables. Note that there are $4$ levels of XOR tables now, since we have $16$ vectors to be XORed together. Also note that the very input and output are {\em not} encoded with $\Enc$.
	
	See Equations~\ref{eq:extencin} and \ref{eq:extencout} for a sketch of the flow through input and output external encoding, respectively. Note that external encoding can be trivially turned off by setting $E$ and $F$ equal to identity.
	
	\begin{align}
	\label{eq:extencin}
		&\xrightarrow{\textnormal{input}}\underbracket{\IMB\circ G\rarr\Enc^{-1}}_{\textnormal{Type I}}\rarr\Bigl(\underbracket{\Enc\rarr\xor\rarr\Enc^{-1}}_{\textnormal{Type IV}}\Bigr)^4\rarr\ldots\\[.5em]
	\label{eq:extencout}
		\ldots&\rarr\underbracket{\Enc\rarr F^{-1}\circ\IMB^{-1}\rarr\Enc^{-1}}_{\textnormal{Type I}}\rarr\Bigl(\underbracket{\Enc\rarr\xor\rarr\Enc^{-1}}_{\textnormal{Type IV}}\Bigr)^3\rarr\underbracket{\Enc\rarr\xor}_{\textnormal{Type IV}}\xrightarrow{\textnormal{output}}
	\end{align}


% ==============================================================================
% ===   K N O W N   A T T A C K S   &   E N H A N C E M E N T S              ===
% ==============================================================================

\subsection{Known Attacks \& Enhancements}
\label{sec:known}

As outlined in Section~\ref{sec:catmouse}, sooner or later after introduction of a new white-box implementation of a cipher somebody usually comes up with an effective attack. This is also the case of WBAES introduced by Chow et al.\ in 2002 as well, it was broken by Billet et al.\ \cite{billet2004cryptanalysis} two years later. The attack is called {\em BGE attack} -- it is an abbreviation of authors' names.

\subsubsection{BGE Attack}
	
	BGE attack is a pure algebraic attack, therefore it requires access to all tables implementing WBAES.
	
	\begin{note}
	\label{note:reverse}
		In a real-world scenario, we first need to reverse-engineer the provided binary code in order to access the tables. This can make any analysis really painful -- such white-box cryptography programs are often further protected against reverse engineering effort with various techniques.
	\end{note}
	
	We will not cover the BGE attack in detail here, it is out of the scope of this thesis, but it motivated further research. On the one hand, the BGE attack has been adapted for a broader class of ciphers by Michiels et al.\ \cite{michiels2008cryptanalysis}, on the other hand, novel white-box approaches have been introduced. Among others (e.g., \cite{michiels2007cryptographic, xiao2009secure}), we focus on Karroumi's contribution \cite{karroumi2010protecting}. Even though it has been shown by Klinec \cite{klinec2013white} to be equivalent to the original WBAES by Chow et al., it will later support the origin of our contribution.

\subsubsection{Karroumi's Approach with Dual Ciphers}
	
	Karroumi noticed that the BGE attack exploits the knowledge of specific AES constants, for instance coefficients inside $\MixColumns$ (see Equation~\ref{eq:mixcolmatr}) or, more importantly, coefficients of affine transformation inside $\SubBytes$ (see Equation~\ref{eq:sbox} and Remark~\ref{rem:sboxaff}). In 2002, Barkan et al.\ \cite{barkan2002many} came up with the idea of changing these coefficients in an appropriate way and Karroumi suggested to use such modified AES as a basis for an upgraded WBAES. In general, such a modified cipher is referred to as a {\em dual cipher}.
	
	Originally there were $240$ dual AES ciphers by Barkan et al., the list has been later extended by Raddum \cite{raddum2004more} to $9\,360$ and further by Biryukov et al.\ \cite{biryukov2003toolbox} to $61\,200$ dual AES ciphers.
	
	Note that there is a simple relation between the AES and a dual AES: for each dual cipher, there exists a linear mapping $\Delta$ that maps a state of the AES on a state of the dual AES and vice versa. Since both plaintext and ciphertext are an AES state at some point, it follows that $P_{dual} = \Delta P$ and $C_{dual} = \Delta C$, moreover the same holds nontrivially for the key, i.e., $K_{dual} = \Delta K$.
	
	Karroumi incorporates $4\cdot 10$ distinct dual AES'es into WBAES (for each column of each round), while appropriate linear mappings $\Delta$, are simply composed with mixing bijections in each round. Hence the overall structure of WBAES remains unchanged. Karroumi presumes that this modification raises the complexity of the BGE attack from originally $2^{30}$ to $2^{91}$, since the attacker seems to need to loop through all four-tuples of dual AES'es and to run the standard BGE with constants of those dual AES'es.
	
	Intuitively, since the transfer to dual cipher is only done by a linear transformation, and since each original WBAES round is wrapped around with a random linear transformation (mixing bijection), it seems that we can consider each original WBAES round as internally using a dual AES and vice versa. Klinec has shown in \cite[Proposition~2]{klinec2013white} that this is indeed true.

\subsubsection{Klinec's Equivalence Result}
	
	Here we state Klinec's result, but note that, according to the original proof, it is rather a consequence of a much stronger property, see below.
	
	\begin{prop}[Klinec]
	\label{thm:dualequiv}
		Karroumi's WBAES scheme can be broken with the BGE attack with the same time complexity as the original WBAES scheme.
	\end{prop}
	
	\begin{proof}
		We refer to \cite{klinec2013white} for the full proof, it is rather long and technical and requires a broader context, too.
	\end{proof}
	
	\begin{remark}
	\label{rem:dualequiv}
		The core observation of Klinec's proof is that equations of Karroumi's WBAES can be rewritten to the form of the original Chow's WBAES, i.e., without using dual ciphers at all. Concretely the transformations to dual cipher and operations in dual cipher are shown to be absorbed by surrounding random bijections (namely $\IMB^{-1}\circ\Enc$ and its inverse, in addition also by $\MB$ and its inverse), yielding the original Chow's WBAES.
	\end{remark}
	
	As a consequence of the previous remark, it follows that the BGE attack breaks Karroumi's WBAES with the same time complexity as the original WBAES scheme -- simply because both is the same from the attacker's perspective.
	
	\begin{note}
	\label{note:dualsbox}
		Later in this thesis, we will attack Klinec's implementation \cite{klinec2013implementation} of WBAES by Chow et al. We will utilize Remark~\ref{rem:dualequiv} in a slightly different way.
		
		Note that, among others, dual AES allows us to alter coefficients of the affine transformation inside $\SubBytes$ step. Since we do not recognize which instance of dual AES has been used in WBAES, we can expect every single one, i.e., we can suppose that those SBoxes use {\em arbitrary invertible affine mapping}.
	\end{note}
	
