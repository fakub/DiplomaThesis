\section{White-Box AES}
\label{sec:wbaes}

In this section we describe White-Box AES (WBAES) as introduced by Chow et al. \cite{chow2003aes} and revised by Muir \cite{muir2013tutorial} (which we highly recommend). Note that it has already been broken by Billet et al. \cite{billet2005cryptanalysis}, see Section \ref{sec:known}.   %!% ref to subsection, def linear?

The core idea of WBAES is based on usage of lookup tables as outlined in Section \ref{sec:aeslookup}, these are further improved in three ways:
\begin{enumerate}
	\item the implementation is turned into fully-table representation i.e. the key is incorporated into tables as well,
	\item tables are wrapped around by appropriately self-vanishing random bijections, \label{item:wrap}
	\item the boundary is extended into the containing application. \label{item:boundary}
\end{enumerate}
Note that Chow et al. introduced also White-Box DES \cite{chow2003des} but it has been broken by Jacob et al. \cite{jacob2003attacking} much earlier than WBAES because it was lacking the feature \ref{item:boundary} from the previous list.

\begin{notion}
\label{notion:table}
	Since WBAES will be a lot about lookup tables, let us introduce a simplifying notion: a table will be told to have dimensions $m\times n$ if it has $m$-bit input values and $n$-bit output values. Note that the real size (i.e. memory consumption) of such table is $2^m\cdot n$ bits since you need $2^m$ entries of size $m$ to treat all possible $m$-bit input values.
\end{notion}


% ======================================================================
% ===   E N C O D I N G S                                            ===
% ======================================================================

\subsection{Encodings}

Let us first define some notion for the feature \ref{item:wrap} from the previous list as defined by Chow et al. \cite{chow2003aes}.

\begin{defn}[Encoding]
	Let $X$ be a transformation from $m$ bits to $n$ bits. Choose an $m$-bit bijection $F$ and an $n$-bit bijection $G$. Call $X' = G \circ X \circ F^{-1}$ an {\em encoded} version of $X$. $F$ is an {\em input encoding} and $G$ is an {\em output encoding}.
\end{defn}

Note that, given a fixed key, AES itself is a $128$-bit bijection which can obviously never be implemented as a lookup table because its size would be $2^{128}\times 128$ bits. (On the other hand, this would be black-box equivalent -- you only know the input/output behavior.) Thus we will rather consider only a special kind of bijections -- we will use a couple of smaller bijections and concatenate their outputs, a definition follows.

\begin{defn}[Concatenated encoding]
\label{def:concat}
	Let $F_i$ be $n_i$-bit bijections for $i=\atob{1}{k}$ and let $n = n_1 + \ldots + n_k$. The {\em function concatenation} $F_1 \| F_2 \| \ldots \| F_k$ is the $n$-bit bijection $F$ such that $F(b) = F_1(\atob{b_1}{b_{n_1}})\| F_2(\atob{b_{n_1+1}}{b_{n_1+n_2}})\|\ldots\| F_k(\atob{b_{n_1+\ldots+n_{k-1}+1}}{b_n})$ for any $n$-bit word $b=b_1b_2\ldots b_n$ where $\|$ denotes word concatenation. Clearly $F^{-1} = F_1^{-1}\| F_2^{-1}\| \ldots \|F_k^{-1}$.
\end{defn}

The following definition introduces the self-vanishing property of the bijections.

\begin{defn}[Networked Encoding]
\label{def:netw}
	A {\em networked encoding} for computing $Y\circ X$ (i.e. transformation $X$ followed by transformation $Y$) is an encoding of the form
	\[
		Y'\circ X' = (H\circ Y\circ G^{-1})\circ(G\circ X\circ F^{-1}) = H\circ(Y\circ X)\circ F^{-1}
	\]
	where $F$, $G$, $H$ are bijections of appropriate size, $F$ and $H$ are {\em external encodings} (input and output encoding, respectively) and $G$ is an {\em internal encoding}.
\end{defn}

Note that, given only tables for $Y'$ and $X'$, $G$ can be totally forgotten to be able to compute $Y\circ X$ provided that you know the input and output encoding. Such networked encoding can be obviously applied to much longer composition of transformations and you still only need to know the input and output encoding.   %!% někam popsat I/O encoding


% ======================================================================
% ===   R E O R D E R E D   A E S                                    ===
% ======================================================================

\subsection{Reorded AES}

In order to begin with fully-table representation, we need to reorder AES operations so that the resulting tables can be easily composed; let us first have a brief look at AES algorithm \ref{alg:aes}. Note that we can ``move'' the forcycle one row upwards with appropriate shift of indexes in $\ExpandedKey$. Then we switch $\ShiftRows$ and $\SubBytes$ without any side-effect since $\SubBytes$ operates byte-wise. Finally we switch $\ShiftRows$ and $\AddRoundKey$ while we moreover have to apply $\ShiftRows$ on $\ExpandedKey$ so that we get equivalent algorithm. The modified version is as follows:
\begin{alg}
\label{alg:reordaes}
	Reordered AES encryption.
	\begin{algorithmic}[1]
		\Function{Reordered\_AES\_Encryption}{$Plaintext,Key$}
			\State $\ExpandedKey \gets \KeySchedule(Key)$
			\State $State \gets Plaintext$
			\For{$Round = 0 \to 8$}
				\State $\ShiftRows(State)$
				\State $\AddRoundKey(State, \ShiftRows(\ExpandedKey[Round]))$
				\State $\SubBytes(State)$
				\State $\MixColumns(State)$
			\EndFor
			\State $\ShiftRows(State)$
			\State $\AddRoundKey(State, \ShiftRows(\ExpandedKey[9]))$
			\State $\SubBytes(State)$
			\State $\AddRoundKey(State, \ExpandedKey[10])$
			\State\Return $State$
		\EndFunction
	\end{algorithmic}
\end{alg}


% ======================================================================
% ===   F U L L Y - T A B L E   R E P R E S E N T A T I O N          ===
% ======================================================================

\subsection{Fully-Table Representation}

As outlined earlier, WBAES construction turns AES operations into table lookups only. Moreover we compose certain tables and wrap them all by internal encodings in the fashion of Definition \ref{def:netw}. WBAES generator thus inputs an AES key and a random seed and outputs key-dependent tables which can serve as a WBAES instance.

\subsubsection{T-Boxes}
	
	First we compose $\AddRoundKey$ with the following $\SubBytes$ step yielding a $\TBox$, also denoted by $T_{i,j}^r$,
	\begin{equation}
		T_{i,j}^r(x) = S(x\xor k_{sr(i,j)}^r) , \qquad i,j=\atob{0}{3},\;r=\atob{0}{8}
	\end{equation}
	where $k_{sr(i,j)}^r$ stands for $\ExpandedKey$'s byte in $r$-th round at position $(i,j)$ shifted by $\ShiftRows$ (here denoted by $sr(i,j)$). The last round is treated separately as
	\begin{equation}
		T_{i,j}^{9}(x) = S(x\xor k_{sr(i,j)}^9) \xor k_{i,j}^{10} , \qquad i,j=\atob{0}{3} .
	\end{equation}
	$\TBox$ can be implemented as $8\times 8$\footnote{According to Notion \ref{notion:table}.} lookup table and there are $16\cdot 10 = 160$ of them.

\subsubsection{MixColumns}
	
	$\MixColumns$ (here $\MC$) can be turned into four $8\times 32$ lookup tables followed by three XOR operations as outlined in Section \ref{sec:aeslookup}. The difference from classical implementation is that the following XOR operations must be turned into lookup tables as well.
	
	Since the input size of such XOR operation would be $2\cdot 32$ bits, we first split inputs into smaller $4$-bit segments which can be treated separately. In the end we simply concatenate them obtaining the final result. Such smaller XOR table is then $2\cdot 4\times 4$ and you need to perform $8\cdot 3=24$ such lookups per one column.

\subsubsection{ShiftRows}
	
	$\ShiftRows$ does not need table representation since it only moves bytes. We will see later that it can be implemented correctly.


% ======================================================================
% ===   W B A E S   C O N S T R U C T I O N                          ===
% ======================================================================

\subsection{WBAES Construction}

So far we can represent the whole AES by lookup tables only. Now we will introduce a diffusion feature called {\em Mixing Bijection}, compose certain tables together and insert encodings which are actually a confusion feature.

\subsubsection{Mixing Bijection}
	
	Mixing bijection ($\MB$) is a random column-wise (i.e. $32$-bit) linear bijection inserted after $\MC$ and inverted in a separate step. Since it inputs the output of $\MC$, it can be composed together into $\MB\circ\MC$. The table implementation is then created in very the same manner as for $\MC$ itself.
	
	%!% 4x4 submatrices of full rank !!!

\subsubsection{Input Mixing Bijection}
	
	Input mixing bijection ($\IMB$) is a byte-wise variant of mixing bijection. It provides diffusion after inverting mixing bijection and before entering next $\TBox$. Since it is a linear mapping, it can be composed with inverted mixing bijection and since it is a byte-wise mapping it can be composed with $\TBox$ as well. The reason for this smaller diffusion element will be given later.

\subsubsection{Table Composition}
	
	Note that $\TBox$ is implemented as an $8\times 8$ table which is followed by an $8\times 32$ table of $\MB\circ\MC$. Therefore it is reasonable to compose those appropriately subsequent tables together -- we save both space and time.

\subsubsection{Inserting Encodings}
	
	Note that the (partial) outputs of linear operations are first split into $4$-bit blocks and then two independent $4$-bit words are XORed together yielding $4$-bit blocks. Thus we can only use $4$-bit internal encodings ($\Enc$) on both sides of each XOR. This is also the reason why input mixing bijection has been introduced -- it diffuses two $4$-bit blocks together before $4$-bit encoding is applied.
	
	Let us demonstrate previous ideas in a single sketch of the flow during one inner round (i.e. not the first nor the last) showing all operations, table compositions, mixing bijections and encodings. Tables are given their names as introduced by Chow et al. \cite{chow2003aes}.
	\begin{align}
		\ldots&\rarr\underbracket{\Enc\rarr\IMB^{-1}\xrightarrow{\textnormal{plain}}\TBox\xrightarrow{\textnormal{plain}}\MB\circ\MC\rarr\Enc^{-1}}_{\textnormal{Type II}}\rarr\Bigl(\underbracket{\Enc\rarr\xor\rarr\Enc^{-1}}_{\textnormal{Type IV}}\Bigr)^2\rarr\nonumber\\[.5em]
		&\rarr\underbracket{\Enc\rarr\IMB\circ\MB^{-1}}_{\textnormal{Type III}}\rarr\Bigl(\underbracket{\Enc\rarr\xor\rarr\Enc^{-1}}_{\textnormal{Type IV}}\Bigr)^2\rarr\ldots
	\end{align}
	
	%!% připojit Muirovo obrázek