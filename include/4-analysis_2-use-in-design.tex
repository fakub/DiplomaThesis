\section{Use in White-Box Crypto Design}
\label{sec:useindesign}

Note that the attack allows us to reveal concrete physical addresses which contributed to the key recovery. Our intention was to find the corresponding piece of source code and later possibly also the value which is transformed into physical address -- it would most likely be an index in an array. We tried several approaches how to find that place in source code, see the following list.

\begin{description}
	\item[Watchpoint in GDB.]
		We set up a watchpoint in GDB running in the same terminal session while watching an address previously caught by PIN. Probably since PIN uses different environment, GDB did not catch the address at all.
		%!% zkusit valgrind a pak GDB?
	\item[GDB connected to PIN debugging interface.]
		We set up a connection between PIN application debugging interface and GDB as described in PIN manual \cite{pin214manual}, but could not catch {\em any} address. GDB was complaining that it cannot insert hardware breakpoint, on the other hand, software breakpoints did not work either.
		%!% vyzkoušet ten příklad z manuálu, jestli to jenom blbě nekompiluju ...
	\item[Outputting all array indexes.]
		Note that array indexes are very likely the values which are transformed to leaking physical addresses. Hence it should be possible to output and attack all indexes used within the encryption procedure. But, surprisingly, we did not succeed breaking the key from such ``traces''.
	%!% vyzkoušet debugger CLionu !!!
\end{description}

Each of these approaches appeared to bring valuable information about where the key leaks. The leakage position could be later used to identify a specific table access which leaks, hence it could possibly help us to clarify what is the actual reason why this attack works.

Note that, to the best of our knowledge, there is no explanation why this attack is actually possible. It seems impossible since all of the AES intermediate results we attack are hidden by internal encodings, in fact random bijections. Note that a random bijection is extremely unlikely to provide any correlation between any input and output bit.

Even though it is not fully understood, the benefit of this attack is obvious -- it introduces a principially different approach to break white-box crypto implementations. This attack may help to address weaknesses of a future white-box crypto implementation and possibly increase its resistancy to this kind of attack.

\subsection{Countermeasures against DCA}

Since DCA is based on an algorithm which was originally developed for physical measurements of certain hardware emissions, we can look for some inspiration back in hardware model. There are several countermeasures, see the following list for some of them (basic ones are introduced in \cite{chari1999towards,goubin1999des}).
\begin{description}
	\item[Masking with random values.] Cryptographic hardware can run some unpretendable source of random data and thus mess up values in traces (i.e.\ providing a stronger noise). However, there is no noise in white-box context.
	\item[Reordering instructions.] Remind that our algorithms required aligned traces, therefore it would be fatal if the leaking position was at several different places. However, this could be possibly overcome in white-box context since we have a full control over the execution. We can achieve this either by reverse engineering, or, much easier, by aligning traces with respect to instruction address trace.
	\item[Adding random delays.] Note that random delays have very similar effect as the previous countermeasure. Moreover we could possibly control the random source and simply keep it constant.
\end{description}
In general, we cannot rely on any source of {\em dynamic} random data (generated during program execution). All we need to rely on is {\em static} random data (generated during instantiation). On the other hand, dynamic randomness can be used in a white-box implementation -- just as another level of ``obfuscation''.

Bos et al.\ further propose to use some ideas from {\em threshold implementations} \cite{nikova2006threshold} or use of external encodings -- here they emphasize that more research is needed.
