\section{Explanation Attempt}
\label{sec:attempt}

First of all, let us remind the flow of information through WBAES tables (originally in Equation \ref{eq:wbaesflow}), especially its first part.
\begin{equation}
\label{eq:wbaesfirst}
	\ldots \rarr \underbracket{\Enc \rarr \IMB^{-1} \xrightarrow{\textnormal{plain}} \TBox \xrightarrow{\textnormal{plain}} \MB \circ \MC \rarr \Enc^{-1}}_{\textnormal{in table}} \rarr \ldots
\end{equation}
Note that the last plain AES state (denoted with ``plain'' in the previous equation) is actually an $8$-bit output of the first SBox i.e.\ the very original target. Let us see what happens next: a composed linear mapping $\MB\circ\MC$ is applied, resulting in a $32$-bit output, and then each $4$-bit nibble is passed through a $4$-bit random bijection $\Enc^{-1}$. 

Note that $\Enc$ is only a $4$-bit bijection 

%!% attempt to explain: výstup SBoxu je problitej náhodnym lin. zobrazenim tj. prvky výstupu jsou zase jenom skal. součinem s nějakým náhodnym nenulovym vektorem ... po čtveřicích jsou takový bity problitý náhodnou bijekcí (právě že jenom po 4!)
% nikdo přece netestuje nelinearitu/nekorelovanost techdle bijekcí ... neni to třeba tak že každá 3. koreluje na nákym výstupnim bitu s nákym vstupnim bitem? nebo klíďo jejich lin. kombinací?
% ale mam pocit že to jde ověřit právě výpisem mezivýsledků a útokem na ně, co si pomatuju tak to nešlo ...

% ze dřívějška:
% neni třeba problém v tom že to jsou concatenated encodings? i.e. ze 2x4 bitů postavený 8 bitový bijekce
% jesli takovejch už třeba neni moc málo nebo jestli náhodou nedávaj docela často nákou korelaci
% třeba zkusit dokázat, že random bijection ani náhodou nezanechá korelaci v nákym bitu, mělo by tam bejt negligible
% asi se trochu rozepsat a odůvodnit ...
% třeba to, jak jsou ty bijekce generovaný, že to je z nákýho pejpru, kterej něco ukazuje
