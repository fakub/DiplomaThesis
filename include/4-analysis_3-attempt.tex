\section{Explanation Attempt}
\label{sec:attempt}

First of all, let us remind the information flow through WBAES tables (originally in Equation \ref{eq:wbaesflow}), especially its first table.
\begin{equation}
\label{eq:wbaesfirst}
	\ldots \rarr \underbracket{\Enc \rarr \IMB^{-1} \xrightarrow{\textnormal{plain}} \TBox \xrightarrow{\textnormal{plain}} \MB \circ \MC \rarr \Enc^{-1}}_{\textnormal{in table}} \rarr \ldots
\end{equation}
Note that the last plain AES state (denoted with ``plain'' in the previous equation) is actually an $8$-bit output of the first SBox i.e.\ the very original target. Let us see what happens next: a composed linear mapping $\MB\circ\MC$ is applied resulting in a $32$-bit output (using an idea outlined in Section \ref{sec:aeslookup}), then each $4$-bit nibble is passed through a $4$-bit random bijection $\Enc^{-1}$ (further simply $\Enc$).

\begin{description}
	\item[$\MB\circ\MC$.] We can view each bit $t$ of the $32$-bit output of $\MB\circ\MC$ as a scalar product of a row $[R]^T$ of matrix representing $\MB\circ\MC$ (i.e.\ a random-like\footnote{Cannot be zero (follows from Note \ref{note:fullrank}).} $8$-bit vector) with its $8$-bit input $[S]$ which is actually an output of the first SBox. We get
	\begin{equation}
		t = [R]^T\cdot [S] = [R]^T\cdot\Bigl( \{\texttt{1f}\}\cdot\bigl([K]+[P]\bigr)' + [\texttt{63}] \Bigr) ,
	\end{equation}
	where $\{\cdot\}$ stands for respective binary matrix of multiplication modulo $x^8+1$ and $(\cdot)'$ for Rijndael inverse (i.e.\ there is the output of the first SBox in the big parantheses, cf.\ Equation \ref{eq:sbox}). This equation can be easily turned into the form of previously used target (Equation \ref{eq:tba}), indeed
	\begin{align*}
		[R]^T\cdot\Bigl( \{\texttt{1f}\}\cdot\bigl([K]+[P]\bigr)' + [\texttt{63}] \Bigr) &= \bigl([R]^T\cdot \{\texttt{1f}\}\bigr)\cdot\bigl([K]+[P]\bigr)' + \bigl([R]^T\cdot [\texttt{63}]\bigr) = \\
		&= [\bar R]^T \cdot \bigl([K]+[P]\bigr)' + \bar r ,
	\end{align*}
	where $\bar R$ is just a different random-like vector, which plays the role of $B$ in Equation \ref{eq:tba}, and $\bar r$ is a constant bit. Remind that additive constant does not need to be considered inside target (see Remark \ref{rem:pqeffect}). It follows that each of these $32$ bits perfectly matches some of our $255$ targets introduced in Section \ref{sec:unify}.
	%!% no diffusion at the moment! depends on single byte
	
	If we could observe these intermediate results, we would get the difference of means equal to $1.0$ (cf.\ attack against unprotected implementation in Section \ref{sec:naiveaes}).
\end{description}
\begin{remark}
\label{rem:enc}
	It follows that the protection against our $255$ targets is fully and solely accomplished by $\Enc$.
\end{remark}
\begin{description}
	\item[$\Enc$.] Remind that $\Enc$ is a (only) $4$-bit random bijection. Note that, unlike confusion elements in regular ciphers (e.g.\ SBox in AES), there is no non-linearity check\footnote{Linearity in ciphers is studied by {\em linear cryptanalysis}, see e.g.\ \cite{matsui1993linear} for detail.} in WBAES by Chow et al.\ \cite{chow2002aes}.
	
	The choice of $\Enc$ actually only relies on the fact that the ratio of affine mappings among all random bijections is extremely low (less than $0.000\,002\%$ for $4$-bit bijections). This justification is very poor since it does not even address the case where only one output bit of $\Enc$ is an affine mapping of the input!
	
	There are actually quite many bijections which are affine in single output bit. Indeed, there are $2\cdot4\cdot(2^4-1)\cdot8!\cdot8! \approx 2\cdot10^{11}$ of them among $16! \approx 2\cdot10^{13}$ bijections on $4$-bits altogether which already makes some $1\%$ ratio! Also note that in such case, the maximum difference of means would be still $1.0$ since this affine mapping could be composed with the previous affine mappings yielding an affine mapping again.
	
	But we do not necessarily need a fully affine mapping at some bit, it would be sufficient if the mapping were affine for most inputs. Clearly, there are much more such mappings. For this reason, this intermediate result of WBAES seems to be most likely that which causes the leakage.
\end{description}

To support our hypothesis, we modified the source code of {\tt KlinecWBAES} to output these intermediate products (i.e.\ outputs of the first Type II tables, see Equation \ref{eq:wbaesflow}) and ran the attack with these values instead of memory traces. The attack indeed succeeded, which fully supports our explanation.

%!% v čistých mezivýsledcích se nevyskytuje neuniformní rozložení leak bitů

% perla (jakože fakt nebrat vážně): Note that a random bijection is extremely unlikely to provide any correlation between any input and output bit.
