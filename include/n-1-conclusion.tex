% cleardoublepage and phantomsection force that link jumps ABOVE heading, normally it would jump UNDER heading
\cleardoublepage\phantomsection

\addcontentsline{toc}{chapter}{Conclusion}
\chapter*{Conclusion}
\markboth{\MakeUppercase{Conclusion}}{}
\label{chap:concl}

After a general introduction to cryptography, we presented three attack contexts: black-box, gray-box and white-box, with an emphasis on the last-named. Then we described another pillar of this thesis -- the Advanced Encryption Standard (AES, \cite{fips2001aes}) and its white-box variant by Chow et al.\ (WBAES, \cite{chow2002aes}).

Next we presented a side-channel attack that seems to be unrelated to the white-box attack context. However, we also described their relationship as originally outlined by Bos et al.\ \cite{bos2015differential}, and provided a detailed description of this novel attack. Here we particularly emphasized one practical consequence -- no reverse engineering effort is needed for this attack.

In Chapter~\ref{chap:results}, we first reproduced results of the attack by Bos et al.\ against Klinec's implementation \cite{klinec2013implementation} of WBAES. Then we extended this attack from $16$ to $255$ targets, and finally we introduced a simple unifying approach for our contributions.

In the fourth chapter, we analyzed the attack from three perspectives. Firstly, we suggested a blind attack (i.e., an attack without knowing the actual key) based on our observations. Secondly, we commented on the usage of the attack in a white-box cryptography design and presented possible practical countermeasures that, however, did not seem to avoid this attack. Thirdly, we analytically identified and experimentally confirmed which intermediate result leaks information. We also outlined which is the only building block of WBAES that attempts to protect it from our enhanced attack, and why its choice is inappropriate.

We concluded this thesis in Chapter \ref{chap:future}, where we summarized possible topics of further research.
