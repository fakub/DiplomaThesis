\section{Enhancements to the Attack}   %!% to/of ?

Since Chow is eqivalent to Karroumi according to Klinec, Bos et al. came up with the idea of attacking different target instead of the output of $\SubBytes$. Let us remind the definition of $\SubBytes$ (originally defined in Equation \ref{eq:sbox}),

\begin{equation}
\label{eq:sbox2}
	S(A) = (x^4 + x^3 + x^2 + x + 1)A' + x^6 + x^5 + x + 1 \pmod{x^8+1} .
\end{equation}

Note that it is an affine mapping of its input's inverse, indeed it holds for $P = x^4 + x^3 + x^2 + x + 1$ and $Q = x^6 + x^5 + x + 1$ that $S(A) = P\cdot A' + Q$. Now remind that Karroumi's approach changes the original affine mapping in $\SubBytes$ to a different one.

So the idea was not to attack the output of $\SubBytes$ which is an affine mapping applied to input's inverse but attack the inverse itself. Surprisingly, this recovers the key even faster!

%!% asi jsem to z mailu Teuwena pochopil: žádnej SBox ale jenom inverze inputu i.e. $(k\oplus m)^{-1}$

Idea of Bos et al.: use multiplicative inverse only instead of full $SBox$, see Equation \ref{eq:sbox}. Hence we get $(k\oplus m)^{-1}$ as a new attack target.

% psal já / Teuwen:
%~ > I only wonder about the reasoning why Karroumi is more than Chow since
%~ > it seems to have been shown to be equal (based on what I wrote in my
%~ > previous email).
%~ 
%~ Well I've no problem to break completely Chow with standard DCA so there
%~ is something a bit more in Karroumi. Obviously not enough to make it
%~ robust enough...

%!% nápad: podle toho co vidim v memtrace, neni lepší útočit na 0x........??..; na 0x.........??. nebo dokonce na 0x........???? místo 0x..........??
% zkus vimdiff 0.txt 1.txt v attack/enhancements
% to je ale asi uplná sračka

