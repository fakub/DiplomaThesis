\section{White-Box Cryptography}
\label{sec:wbc}

As we already mentioned, classical ciphers are designed with respect to the black-box attack context, and even under this assumption, it has not been theoretically shown that there exists a secure one; see Section~\ref{sec:provable}. We also presented gray-box and white-box attack contexts in Section~\ref{sec:bbtowb} with a short reasoning of the gray-box model.

Now let us have a closer look at WBAC. As we outlined previously, the attacker is very powerful. And even though she is so powerful, she shall not be able to, besides recovering the key, turn an encryption procedure into decryption one and vice versa.

Let us now justify why we are introducing so extreme case as WBAC. Probably the most intiutive motivation stems from the situation, where a malicious software ``lives'' in a device together with some software, which is intended to use confidential cryptographic material. For instance, we could use third-party keys (especially in DRM\footnote{Digital Rights Management.}). Note that another motivation will be given after we introduce required concepts (in Note~\ref{note:motiv}).

In many use cases, public key cryptography appears to work as well. However, public key (asymmetric) ciphers are far slower than symmetric ones, hence useless in many scenarios. Note that there is no known white-box variant of a public key cipher.

There were several attempts to introduce a ``secure'' white-box variant of a known cipher. However, most of today's commercial WBAC resistant solutions rather disobey the Kerckhoffs' principle, and rely on security by obscurity and tons of software obfuscation and anti-reverse engineering techniques.

%?% Motivation for white-box crypto, white-box crypto history, milestones   % Bos et al. to hezky shrnul


% ==============================================================================
% ===   P R O G R A M   O B F U S C A T I O N                                ===
% ==============================================================================

\subsection{Program Obfuscation}
\label{sec:impos}

The field of white-box cryptography is closely related to program obfuscation, which was pioneered by Barak et al.\ \cite{barak2001possibility}. Let us first define what a program obfuscator is, then we give some consequences and present a very interesting theoretical result of their paper.

\subsubsection{Program Obfuscator}
	
	Informally, a {\em program obfuscator} inputs a description of an algorithm (e.g., an input of universal Turing machine or a C code), and transforms it into a functionally equivalent obfuscated algorithm description. Such an obfuscated program should prevent any effective adversary from learning anything she could not learn if she only had an oracle access to the program.
	
	\begin{defn}[Program Obfuscator; taken from \cite{barak2001possibility}, simplified]
	\label{def:obfus}
		{\em Program obfuscator} is an effective (possibly randomized) algorithm, denoted by $O$, if the following three conditions hold:
		\begin{enumerate}
			\item {\em functionality:} for every encoding $C$ of an algorithm $A_C$, $O(C)$ encodes an algorithm which outputs the same as $A_C$ on each input,
			\item {\em polynomial slowdown:} for every encoding $C$ of an algorithm $A_C$, the length and the running time of $O(C)$ is at most polynomially larger than that of $C$,
			\item {\em virtual black-box property:} for every effective algorithm $B$, there exists an effective adversary $A$ such that for every encoding $C$ of an algorithm $A_C$ it holds that
			\[
				\biggl| \Pr\Bigl(B\bigl(O(C)\bigr)=1\Bigr) - \Pr\Bigl(A({\cal O}_{A_C})=1\Bigr) \biggr| \
			\]
			is negligible in terms of $|C|$.
		\end{enumerate}
	\end{defn}
	
	In other words, the virtual black-box property states that there is no effective algorithm, which would be able to gain more information from an obfuscated code than any adversary with an oracle access only, hence black-box property.

\subsubsection{Consequences in Cryptography}
	
	Besides turning a cipher into a WBAC resistant variant and several applications in software protection, a program obfuscator would allow several other exotic cryptographic constructions as well. As an example, let us mention Fully Homomorphic Encryption (also called {\em the holy grail of cryptography}, introduced in 1978 by Rivest et al.\ \cite{rivest1978data}) or transformation of a private key encryption scheme into a public key one.

\subsubsection{Main Result of Barak et al.}
	
	However, the main result of Barak et al.\ basically claims that there is no algorithm that would meet the criteria from Definition~\ref{def:obfus}, i.e., {\em no generic program obfuscator exists}. Although this result might sound disappointedly, it originally claims that there exists {\em certain class} of algorithms which are unobfuscable. Hence there is no evidence that {\em any specific} algorithm is unobfuscable. This gives us the hope that there could be some algorithm implementing a block cipher which is not in this class.

\subsubsection{Going Beyond}
	
	However, the requirements on program obfuscator might be too strong (already suggested by Barak et al.). Indeed, especially in white-box cryptography context, we can weaken the virtual black-box property, since we probably do not care about any information, which does not effectively lead to key recovery, even though we would not learn this information having an oracle access only.
	
	Note that since then, there has been a lot of research using various assumptions, sometimes surprisingly with positive results, too. Most interesting results can be found in \cite{barak2014protecting, brakerski2014virtual}.


% ==============================================================================
% ===   C A T   A N D   M O U S E   G A M E                                  ===
% ==============================================================================

\subsection{Cat And Mouse Game}
\label{sec:catmouse}

It has been conjectured that each white-box resistant implementations gets -- sooner or later -- broken.
\begin{conj}[Chow et al.\ \cite{chow2002aes}]
	No perfect long-term defense against white-box attacks exists.
\end{conj}
Therefore it reminds a cat and mouse game. In this thesis, we present both -- a white-box resistant implementation and actually two attacks against it. Note that there are another examples from the past, e.g., \cite{chow2002des} broken by \cite{jacob2002attacking}.
