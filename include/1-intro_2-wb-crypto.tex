% previously: Finally there is the most extreme case where the attacker has full control over the execution environment -- the white-box attack context. Full control means that the attacker can step the algorithm and directly observe and alter byproducts in the memory or even skip, alter and insert instructions. Clearly, the key cannot be kept in its original form, it must be somehow hidden but still usable for encryption.

\section{White-Box Cryptography}
	
	
	
%~ detailně popsat kde se vzala vč. citací, vývoj, hra na kočku a myš, impos. results
%~ že neni žádnej public key wb

% je to de facto threat model
%~ straightforward AES implementation is totally insecure if certain byproducts or even some information related to them is revealed. An attack exploiting such information leaks is covered in detail in Section \ref{sec:side}.

% implementation can help, e.g. timing attacks
% RSA, an asymmetric cipher, could serve as a good example
% black box: like an oracle, everything holds. ciphers designed wrt black box, why wb? what it is? ...

%~ diffusion vs. confusion
