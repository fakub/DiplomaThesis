\section{Introduction to Cryptography}
\label{sec:introcrypto}

% goal: introduce crypto in a broader context, yet not too much in detail
% starý jako lidstvo samo, bla bla, trochu ukázky z historie a jak to bylo špatný a jak se to umí lámat, security by obscurity, využívá cool matiku
% co po crypto vlastně chci
%~ effective algorithm, uniformly random: $k\xleftarrow{R}K$.
%~ crypto = basis for security mechanisms
%~ CONFIDENTIALITY and much more (e.g. integrity, digital signature, anonymity) or magic: zero knowledge proof of knowledge, FHE
%~  or a complicated construction of crypto primitives (e.g. digital currency, electronic election, electronic auction)
%~ 1. threat model (WB vs. BB later)
%~ 2. propose sth
%~ 3. prove that breaking it would break an underlying hard problem
%~ security by obscurity, on the other hand the useful approach

% bla bla ...
\ldots\\
Let us imagine a typical situation where Alice wants to communicate with Bob confidentially over an insecure channel while Eve may eavesdrop on the channel. In the beginning, let us suppose that Alice and Bob share certain portion of secret information referred to as {\em secret key} and both control a secure execution environment where they intend to run their ciphering algorithm so that Eve can only observe what they send over the channel.

\begin{note}
\label{note:threat}
	The list of such attacker's abilities is referred to as {\em threat model}. Note that in the previous threat model the attacker is quite weak -- she can only read the traffic. If she were for instance able to manipulate the traffic, Alice and Bob would need to introduce another features to their protocol so that they could detect forged messages.
	
	Before we start introducing any means of security, we always need to identify the threat model first. For now we will only consider eavesdropping.   %!% no hrůza
\end{note}

Our first intuitive requirements on Alice and Bob's ciphering algorithm could be expressed as follows: no matter what plaintext Alice and Bob encrypt and send over the channel, Eve shall {\em not} be able to
\begin{enumerate}
	\item recover any portion of any plaintext, \label{item:plainrecov}
	\item recover the secret key. \label{item:keyrecov}
\end{enumerate}
The item \ref{item:plainrecov} requirement is also referred to as {\em confidentiality}. Note that key recovery would lead to plaintext recovery as well, on the other hand not necessarily the other way around. Let us give an example of a primitive cipher.

\begin{example}
\label{ex:shift}
	{\em Shift cipher} is an ancient cipher used already by Julius Caesar and it works as follows: it inputs a string of letters from a finite alphabet $\cal A$, maps its letters bijectively to the numbers from $0$ to $|{\cal A}|-1$ and adds a constant $c\in\{\atob{0}{|{\cal A}|-1}\}$ modulo $|\cal A|$ to each number. Finally it maps the numbers back to letters yielding a string ``shifted'' by $c$ where $c$ can be interpreted as an element of $\cal A$ as well. Decryption can be done simply by substracting $c$ from the ciphertext in a similar way.
	
	Shift cipher can be easily broken by pen and paper only since there are as many keys as letters in the alphabet, typically $26$ english letters. You simply try them all until the plaintext makes sense. Such approach is referred to as {\em brute force attack} or {\em exhaustive search}.
\end{example}
\nomenclature{$\mid S\mid$}{Cardinality of the set $S$}

As we could see in the previous example, shift cipher is not much secure cipher. Therefore we would like to formalize the concept of cipher and its security.

% ======================================================================
% ===   S Y M M E T R I C   A N D   A S Y M M E T R I C   C I P H.   ===
% ======================================================================

\subsection{Symmetric and Asymmetric Cipher}

There are two large families of ciphers -- symmetric and asymmetric. Shift cipher from Example \ref{ex:shift} belongs to the symmetric family which follows from the fact that it uses the same key for both encryption and decryption while asymmetric ciphers use distinct keys for encryption and decryption. Asymmetric ciphers are also referred to as public key ciphers but we will only consider symmetric ciphers in this thesis, a definition follows.

\begin{defn}[Symmetric Cipher]
\label{def:symcipher}
	Let $K$ denote the key space, $M$ the message space and $C$ the ciphertext space. {\em Symmetric cipher} is a pair of efficiently computable (possibly randomized) functions $(E,D)$, $E:K\times M\rarr C$, $D:K\times C\rarr M$, such that $\forall m\in M, k\in K$ it holds
	\begin{equation}
		%~ \Pr\left(D(k,E(k,m))=m\right) = 1 . \label{eq:correct}
		\Pr\left(D(E(l,k),k)=l\right) = 1 . \label{eq:correct}
	\end{equation}
	%?% \footnote{In some use cases it might be useful to require only so called {\em overwhelming} probability; not needed in this thesis.}
\end{defn}
\nomenclature{$\Pr(\omega)$}{Probability of the event $\omega$}

\begin{note}
	Going back to Example \ref{ex:shift}, the key space is $\{\atob{0}{|{\cal A}|-1}\}\sim\cal A$, the message and ciphertext spaces are ${\cal A}^*$ which stands for the set of all strings over the alphabet $\cal A$.
\end{note}

According to the previous definition, we do not seem to have a definition of a proper cipher at all -- the only requirement is corectness (cf. Equation \ref{eq:correct}) -- but it totally lacks any security requirement. As we will see later, it is actually quite complicated to define ``security'' appropriately.

% ======================================================================
% ===   I N F O R M A T I O N - T H E O R E T I C   A P P R O A C H  ===
% ======================================================================

\subsection{Information-Theoretic Approach}

Shannon's groundbreaking work \cite{shannon1949mathematical} on the mathematics of communication gives us a tool -- information-theoretic approach. We can then define secure cipher as a cipher which outputs a ciphertext which carries no information about respective plaintext. Note that this can be rewritten in a more friendly form, a definition follows.

\begin{defn}[Perfectly Secure Cipher]
\label{def:perfsec}
	Let $(E,D)$ be a symmetric cipher. Then it is called {\em perfectly secure} if $\forall m_0,m_1\in M$ such that $|m_0| = |m_1|$ and $\forall c\in C$ it holds $\Pr\left(E(k,m_0)=c\right) = \Pr\left(E(k,m_1)=c\right)$ where $k\unirand K$.
\end{defn}

\begin{note}
\label{note:indist}
	In other words, the attacker is not able to {\em distinguish} encryption of $m_0$ from encryption of $m_1$ by {\em any} means.
\end{note}

There is a cipher which is perfectly secure, see the following example.

\begin{example}[Vernam cipher]
	Let $K = M = C = \{0,1\}^n$ and $k\unirand K$. Given a plaintext $m$, {\em Vernam cipher} applies bitwise XOR of the key $k$ yielding ciphertext $c$ i.e. $c = m\xor k$.
\end{example}

\begin{note}
	Vernam cipher is also referred to as {\em One Time Pad} (OTP). Note that it is crucial that each key is only used once otherwise you can attack the cipher easily: given two ciphertexts using the same key $c_0=m_0\xor k$ and $c_1=m_1\xor k$, one can compute $c_0\xor c_1 = m_0\xor m_1$. This can be practically attacked using some apriori knowledge about plaintexts (e.g. encoding (ASCII), format and language of plaintext (XML and english) and so on) which reduces the amount of possibilities and typically leads to complete plaintext recovery.
\end{note}

Perfect security seems to be exactly what we want from a secure cipher but the following theorem shows that it implies certain inconveniences.

\begin{thm}
\label{thm:kgeqm}
	Let $(E,D)$ be a perfect cipher. Then $|K| \geq |M|$.
\end{thm}

\begin{note}
\label{note:intent}
	This consequence of perfect secrecy seems to go against our intention -- we would like to establish a shared secret key of a limited length and then, using the key, be able to encrypt as much information as we want. For this reason we need to weaken our security assumption. Note that previously there was no assumption on attacker's computing power -- the requirement was actually to hide the  plaintext from {\em any} attacker i.e. including an attacker with unlimited computing power.
\end{note}

% ======================================================================
% ===   C O M P L E X I T Y - T H E O R E T I C   A P P R O A C H    ===
% ======================================================================

\subsection{Complexity-Theoretic Approach}

We would like to utilize the fact that the attacker typically does not possess unlimited computing power but can rather solve only polynomial problems. For this reason we need to change our requirement on ciphertext from ``carries no information'' to ``is computationally impossible to gain some information''. As stated in Note \ref{note:indist}, we would like to make impossible to distinguish which plaintext has been encrypted. We present this idea in the following game.

\begin{game}
\label{game:semsec}
	There are two parties -- a {\em challenger} and an {\em adversary}. First, the challenger chooses a random key $k\unirand K$ and a random bit $b\unirand\{0,1\}$. Then the adversary sends two equally long plaintexts $m_0$, $m_1$ of her choice to the challenger. The challenger then encrypts $m_b$ (according to $b$ she has chosen) and sends the resulting ciphertext $c$ back to the adversary.
	
	The aim of the adversary is to distinguish effectively which $b$ has been chosen by the challenger.
\end{game}
%!% a figure of the game

\begin{note}
\label{note:semsec}
	Resistency of a cipher to this kind of attack is referred to as {\em semantic security}.
\end{note}

\subsubsection{Definition of Semantic Security}

Let us introduce some notion first.

\begin{defn}[Negligible Function]
\label{def:neglfunc}
	Let $\epsilon:\N\rarr\R$. $\epsilon$ is called {\em negligible} if $\forall d\in\N$ $\exists \lambda_0\in\N$ such that $\forall \lambda>\lambda_0$ it holds $\epsilon(\lambda)\leq\frac{1}{\lambda^d}$. In the opposite case, $\epsilon$ is called {\em non-negligible}.
\end{defn}
\nomenclature{$\N$}{Positive integers}
\nomenclature{$\R$}{Real numbers}

\begin{defn}[Overwhelming Function]
\label{def:overwh}
	Let $\epsilon:\N\rarr\R$. $\epsilon$ is called {\em overwhelming} if $1-\epsilon$ is negligible.
\end{defn}

\begin{note}
\label{note:neglconst}
	In practice, {\em negligible} is often used with concrete constants as well. $\epsilon<\frac{1}{2^{80}}$ is considered negligible i.e. events with such probability are not considered to occur during our lives, on the other hand $\epsilon>\frac{1}{2^{30}}$ is considered non-negligible i.e. it is likely to observe an event with such probability.
\end{note}

\begin{defn}[Oracle]
\label{def:oracle}
	Let $F:X\rarr Y$. {\em Oracle evaluating $F$} is a device which, given $x\in X$, evaluates and returns $F(x)$ in unit time.
\end{defn}

\begin{defn}[Statistical Test]   %!% security parameter? effective in terms of what?
	Let $F:X\rarr Y$. {\em Statistical test} is a (possibly randomized) effective algorithm $A$ which only has an access to an oracle evaluating $F$ and which outputs $0$ or $1$, denoted by $A(F)$.
\end{defn}

\begin{note}
	Statistical test will be also referred to as {\em adversary}.
\end{note}

\begin{defn}[Advantage]
\label{def:advant}
	Let $\cal F$, $\cal G$ be two distinct subsets of the set of all functions from $X$ to $Y$ and $A$ an adversary. We define {\em advantage} of adversary $A$ as
	\[
		\Adv(A,{\cal F},{\cal G}) = \left| \Pr\limits_{F\unirand {\cal F}}\bigl(A(F)=1\bigr) - \Pr\limits_{G\unirand {\cal G}}\bigl(A(G)=1\bigr) \right| .
	\]
\end{defn}

In other words, advantage tells us how likely adversary $A$ is able to distinguish a random function from one set of functions from a random function from another set based on its input/output behavior only. According to these sets, advantage can be ``customized'' for a specific purpose. Let us define such advantage for an adversary tempting to distinguish which of her plaintexts has been encrypted i.e. tells us how likely she is to win the Game \ref{game:semsec}.

\begin{defn}[Semantic Security Advantage]
\label{def:ssadvant}
	Let $E: K\times M\rarr C$ be an encryption function of a symmetric cipher, $k\unirand K$, ${\cal E}_b = \{E_b:M^2\rarr C \mid E_b(m_0,m_1) = E(k,m_b)\}$ for $b=0,1$ and $A$ an adversary. We define {\em semantic security advantage} of adversary $A$ as
	\[
		\AdvSS(A,E) = \Adv(A,{\cal E}_0,{\cal E}_1) .
	\]
\end{defn}

Semantic security advantage gives us a reasonable meter for formalization of Game \ref{game:semsec}. We use it to define another notion of security, cf.\ Definition \ref{def:perfsec} (Perfectly Secure Cipher).

\begin{defn}[Semantically Secure Cipher]
\label{def:semsec}
	Let $(E,D)$ be a symmetric cipher. Then it is called {\em semantically secure} if for {\em any} adversary $A$ the semantic security advantage $\AdvSS(A,E)$ is negligible.
\end{defn}

%!% přidat někam útok využívající zkrácení plaintextu po komprimaci
% it makes sense to compress -> encrypt but it might be insecure!

Semantically secure cipher therefore cannot be broken by any probabilistic-polynomial attack with overwhelming probability. On the other hand it might be broken by an attacker with unlimited computing power unlike perfectly secure cipher which cannot be broken either, see Note \ref{note:indist}.

The benefit of semantic security over perfect security is that semantic security does not imply any inconvenient consequence analogous to Theorem \ref{thm:kgeqm} (i.e.\ addresses our intention as stated in Note \ref{note:intent}) and practically provides similar security.

Semantic security is usually achieved using smaller building blocks which are required to have certain properties. The following section describes a building block of one possible approach.

% ======================================================================
% ===   B L O C K   C I P H E R                                      ===
% ======================================================================

\subsection{Block cipher}

There are two main approaches how to construct a semantically secure cipher. They make use of
\begin{enumerate}
	\item a {\em stream cipher}, or
	\item a {\em block cipher},
\end{enumerate}
respectively. In this thesis we will only consider block ciphers, a definition follows.

\begin{defn}[Pseudorandom Permutation]
\label{def:prp}
	Let $E: K\times X\rarr Y$. We call $E$ a {\em pseudorandom permutation} (PRP) if there exists an efficient algorithm which evaluates $E$, the function $E(k,\cdot): X\rarr Y$ is a bijection and there exists an efficient algorithm to compute $E^{-1}(k,\cdot)$ for every $k\in K$.
\end{defn}

\begin{note}
	For $X$ and $Y$ being a set of fixed-length strings, e.g. $X = Y = \{0,1\}^n$, pseudorandom permutation is usually referred to as {\em block cipher}. Note that the previous definition is in some sense very similar to Definition \ref{def:symcipher} -- it does not introduce any security requirements either.
\end{note}

Pseudorandom permutations will play the main role from now and $k$ will play the role of their key. In order to achieve semantic security of certain constructions using a PRP, we need the PRP to behave like a truly random bijection. Let us demonstrate a possible approach on a game similar to Game \ref{game:semsec}.

\begin{game}
\label{game:prp}
	There are two parties -- a {\em challenger} and an {\em adversary}. Let $E:K\times X\rarr Y$ be a PRP. First, the challenger chooses a random key $k\unirand K$, a random bijection $F:X\rarr Y$ and a random bit $b\unirand\{0,1\}$. Then the adversary sends $x\in X$ of her choice to the challenger. The challenger then applies $E(k,\cdot)$ or $F$ if $b=0$ or $1$, respectively, and sends the result back to the adversary.
	
	The aim of the adversary is to distinguish effectively which $b$ has been chosen by the challenger.
\end{game}

Before we give a formal definition of a secure PRP, we define adversary's advantage in the previous game in a manner analogous to Definition \ref{def:ssadvant}.

\begin{defn}[PRP Advantage]
\label{def:prpadvant}
	Let $E: K\times X\rarr Y$ be a PRP, $\cal B$ the set of all bijections from $X$ to $Y$ and $A$ an adversary. We define {\em PRP advantage} of adversary $A$ as
	\[
		\AdvPRP(A,E) = \Adv\Bigl(A,\bigl\{E(k,\cdot)\bigm|k\in K\bigr\},{\cal P}\Bigr) .
	\]
\end{defn}

PRP advantage tells us how likely adversary $A$ is able to distinguish a PRP with a random key from a truly random bijection based on its input/output behavior only. The definition of secure PRP follows.

\begin{defn}[Secure PRP]
\label{def:secprp}
	Let $E: K\times X\rarr Y$ be a PRP. Then $E$ is called {\em secure PRP} if for {\em any} adversary $A$ the PRP advantage $\AdvPRP(A,E)$ is negligible.   %!% negligible in terms of what?
\end{defn}

In other words, given a random key, PRP is secure if it is computationaly indistinguishable from a truly random bijection. An example covering most introduced concepts follows.

\begin{example}
	Let $E:\{0,1\}^{128}\times\{0,1\}^{128}\rarr\{0,1\}^{128}$ be a PRP defined as
	\[
		E(k,x) = \bin\bigl((k)_2 + (x)_2 \mod{2^{128}}\bigr)
	\]
	where $\bin(\cdot)$ stands for number's $128$-bit binary representation and $(\cdot)_2$ stands for a number represented by given binary string.
	
	Now let us construct an adversary $A$ as follows: it generates two arbitrary but distinct $x_0,x_1\in\{0,1\}^{128}$ and feeds them to an oracle computing either PRP $E$ or a truly random bijection. The oracle returns $y_0$ and $y_1$, respectively, and the adversary compares values $(y_i)_2 - (x_i)_2 \mod{2^{128}}$ for $i=0,1$ with each other. If these values are equal, $A$ returns $1$, $0$ otherwise.
	
	Note that if the oracle computes PRP $E$ then these values are always equal. Otherwise, i.e. the oracle computes a truly random bijection, the values are distinct with overwhelming probability. Therefore the PRP advantage of such adversary is
	\[
		\AdvPRP(A,E) = \left| \Pr\limits_{k\unirand \{0,1\}^{128}}\Bigl(A\bigl(E(k,\cdot)\bigr)=1\Bigr) - \Pr\limits_{P\unirand {\cal B}}\Bigl(A(P)=1\Bigr) \right| = 1 - \epsilon
	\]
	where $\cal B$ stands for the set of all bijections on $\{0,1\}^{128}$ and $\epsilon$ is negligible.
	
	It follows that such PRP is totally insecure since there exists an attacker with overwhelming advantage but only a negligible advantage is allowed so that PRP is secure.
\end{example}

\begin{note}
	In the previous example, the adversary exploits the knowledge of the cipher's internal structure -- she knows everything but the key. One could have an idea to hide the structure from the attacker but this is actually a {\em very} bad idea. Once your construction leaks, you need to deploy a new ciphering algorithm confidentially but you have no way how to do so.
	
	You shall rather use a secure publicly known cipher and only rely on the key. Once your key is compromised, you simply establish a new key. Also note that your key is only to be protected within your device but the ciphering algorithm would need to be protected globally.
	
	Such approach hiding the design of a cipher is referred to as {\em security by obscurity} and shall be strictly avoided.
\end{note}

% ======================================================================
% ===   P R O V A B L E   S E C U R I T Y   P R O O F S              ===
% ======================================================================

\subsection{Provable Security Proofs}

Now we know what it is a secure PRP (or a secure block cipher) -- it is based on Game \ref{game:prp} where the aim of the adversary is to distinguish a PRP from a truly random bijection. The obvious question is what happens if we put a secure PRP into Game \ref{game:semsec} where the adversary tries distinguish which of his two inputs has been processed by a cipher. Note that PRP is a cipher (cf. Definition \ref{def:symcipher} and \ref{def:prp}) so it makes a sense. Indeed, given a PRP, there exists an effective algorithm evaluating it and since it is invertible, also Equation \ref{eq:correct} holds.

\begin{note}
\label{note:singleaccess}
	A drawback of PRPs is that they cannot be randomized therefore we will need to limit the amount of adversary-challenger queries to one or, equivalently, change the key after each query. If we do not do so then there exists a trivially winning adversary $A$.
	
	Indeed, such adversary picks three distinct plaintexts $m_0$, $m_1$, $m_2$ (well, let us suppose there are at least three plaintexts in $M$) and sends $(m_0,m_1)$ to the challenger while she gets back some ciphertext $c_a$. In her second query, the adversary sends $(m_0,m_2)$ to the challenger and receives another ciphertext, let say $c_b$. Since PRP is a deterministic bijection, she can decide with certainty:
	\begin{itemize}
		\item $b=0$ if $c_a = c_b$, or
		\item $b=1$ if $c_a \neq c_b$
	\end{itemize}
	and win the game with $\AdvSS(A,E) = 1$ where $E$ stands for the PRP.
\end{note}

\begin{note}
\label{note:randomize}
	The previous note also implies that in order to achieve semantic security of a cipher, the cipher {\em must} encrypt the same plaintext differently each time. This can be achieved using randomization or counters, both approaches are practically used.
\end{note}

Let us modify Game \ref{game:semsec} in a manner described within Note \ref{note:singleaccess} for the purpose of testing PRPs.

\begin{game}
\label{game:semsecprp}
	There are two parties -- a {\em challenger} and an {\em adversary}. Let $E:K\times X\rarr Y$ be a PRP. First, the challenger chooses a random bit $b\unirand\{0,1\}$ and before each round also a random key $k\unirand K$. In each round the adversary sends two equally long plaintexts $m_0$, $m_1$ of her choice to the challenger. The challenger then computes $c = E(k,m_b)$ and sends it back to the adversary.
	
	The aim of the adversary is to distinguish effectively which $b$ has been chosen by the challenger.
\end{game}

\begin{note}
\label{note:semsecprp}
	Based on the previous game, we could define {\em PRP semantic security advantage} ($\AdvPRPSS$) and {\em semantically secure PRP} (cf. Game \ref{game:semsec}, Definitions \ref{def:ssadvant} and \ref{def:semsec}). The difference would be that the oracle access should be redefined -- it would either need to be limited to one or the oracle would need to change the function it evaluates. Due to these technical obstacles we do not provide exact definitions.
\end{note}

The following theorem gives a connection between Game \ref{game:prp} and \ref{game:semsecprp}. It claims that given a PRP $E$ winning Game \ref{game:prp} i.e. $E$ is indistinguishable from a truly random bijection, $E$ wins Game \ref{game:semsecprp} as well i.e. no adversary can distinguish encryption of $m_0$ from encryption of $m_1$.

\begin{thm}
\label{thm:semsecprp}
	Let $E: K\times X\rarr Y$ be a secure PRP. Then it is also a semantically secure PRP.
\end{thm}
\begin{proof}
	We show that for every PRP semantic security adversary $A$ there exist PRP security adversaries $B_0$, $B_1$ such that
	\begin{equation}\label{eq:aleqbb}
		\AdvPRPSS(A,E) \leq \AdvPRP(B_0,E) + \AdvPRP(B_1,E) .
	\end{equation}
	The claim then follows -- if $E$ is a secure PRP i.e. $\AdvPRP(B_0,E) + \AdvPRP(B_1,E)$ is negligible then also $\AdvPRPSS(A,E)$ is negligible therefore $E$ is a semantically secure PRP.
	
	Let us first denote $W_b$ the event when adversary $A$ answers $1$ while the true value of the PRP semantic security game is $b$ for $b=0,1$. In this notation it holds
	\begin{equation}\label{eq:prpssae}
		\AdvPRPSS(A,E) = \left| \Pr(W_1) - \Pr(W_0) \right| .
	\end{equation}
	Let us further denote $R_b$ the same event with only difference -- the PRP semantic security game uses a random bijection instead. It holds
	\begin{equation}\label{eq:rr0}
		\left| \Pr(R_1) - \Pr(R_0) \right| = 0
	\end{equation}
	since $b\unirand\{0,1\}$ and both $R_b$ provide a truly random bijection independent from $b$.
	
	Now we construct adversaries $B_0$, $B_1$ distinguishing PRPs from truly random ones while we make use of adversary $A$. Let us construct $B_0$ first. We ask $A$ to provide $m_0$ and $m_1$, then we feed the oracle with $m_0$ obtaining $c$ as a result. Note that either $c=E(k,m_0)$ or $c=F(m_0)$ according to $b$ where $F$ is a truly random bijection. We provide this $c$ to adversary $A$ and output what $A$ outputs. The advantage of adversary $B_0$ is
	\begin{equation}\label{eq:prpb0}
		\AdvPRP(B_0,E) = \left| \Pr(W_0) - \Pr(R_0) \right| .
	\end{equation}
	We construct $B_1$ in a similar manner (i.e.\ choosing $m_1$ instead) and get
	\begin{equation}\label{eq:prpb1}
		\AdvPRP(B_1,E) = \left| \Pr(W_1) - \Pr(R_1) \right| .
	\end{equation}
	Combining Equations \ref{eq:prpssae}, \ref{eq:rr0}, \ref{eq:prpb0} and \ref{eq:prpb1} we get
	\begin{align*}
		\AdvPRPSS(A,E) &= \left| \Pr(W_1) - \Pr(W_0) \right| \leq \\
		&\leq \left| \Pr(W_1) - \Pr(R_1) \right| + \left| \Pr(W_0) - \Pr(R_0) \right| = \\
		&= \AdvPRP(B_1,E) + \AdvPRP(B_0,E)
	\end{align*}
	which finally proves Equation \ref{eq:aleqbb}.
\end{proof}

\begin{note}
	A semantically secure PRP does not need to be a secure PRP. Indeed, let $E: K\times X\rarr Y$ be a secure PRP and define $E'(k,x) = E(k,x)|1$ where $|$ stands for string concatenation. $E'$ is obviously semantically secure (it remains hard to distinguish encryptions of dinstinct plaintexts) but it can be easily distinguished from a truly random bijection since it always ends with $1$. Advantage of such adversary would be $\AdvPRP(A,E') = \nicefrac{1}{2}$ which is clearly non-negligible.
\end{note}

The previous theorem is especially important for the idea behind its proof which is used in many other proofs related to provable-security. In this proof we construct adversaries $B_0$, $B_1$ which internally employ adversary $A$ and then we use advantage of $A$ to estimate advantages of $B_0$ and $B_1$.

% ======================================================================
% ===   P R O V A B L E   V S   R E A L - W O R L D                  ===
% ======================================================================

\subsection{Provable vs. Real-World Security}

Provable security is one aspect, the other is practical construction of a cipher. Note that there is no provably secure block cipher known. If there were, it would actually imply that $\Pclass \subsetneq \NPclass$ which is an open millenium problem. Indeed, if $\Pclass = \NPclass$ and $E$ were a PRP then adversary could, given a certain (polynomial) amount of plaintext-ciphertext pairs $(m_i,c_i)$, look for a key $k$ such that $c_i = E(k,m_i)$ for all $i$. Then she would simply answer ``truly random'' if no such $k$ exists, otherwise she says ``pseudorandom''. Here the ``truly random'' answer is sure while the ``pseudorandom'' answer is just a conjecture but with sufficiently large probability.
\nomenclature{$\Pclass$}{Deterministic polynomial complexity class i.e. the set of decision problems which can be solved by a deterministic Turing machine in polynomial time}
\nomenclature{$\NPclass$}{Non-deterministic polynomial complexity class i.e. the set of decision problems which can be solved by a non-deterministic Turing machine in polynomial time}

Instead, real-world block ciphers are {\em beleived} to be secure based on a simple fact: no known attack is far better than exhaustive search. One such block cipher is called {\em Advanced Encryption Standard} (AES) and will be of top interest in this thesis. AES will be described in detail in Section \ref{sec:aes}.


% ======================================================================
% ===   F R O M   B L A C K - B O X   T O   W H I T E - B O X        ===
% ======================================================================

\subsection{From Black-Box to White-Box Attack Context}

The other central topic of this thesis is {\em white-box attack context}. In order to explain what it is, let us start with {\em black-box attack context}.

\subsubsection{Black-Box}

As we stated in Note \ref{note:threat}, first of all we need to address attacker's abilities -- the threat model. In the black box model, the attacker is given an access to an oracle (see Definition \ref{def:oracle}) evaluating given encryption algorithm $E$ with a random secret key $k$ while the (primary) goal of the attacker is to recover the key. The attacker has absolutely no information but inputs and respective outputs, she cannot observe any internal routines of encryption, hence black box.

\begin{note}
	There could be many other goals than just key recovery, e.g. guessing the ciphertext in advance or be able to decrypt certain/given ciphertext with non-negligible probability and so on. Actually whatever you can imagine which could be in principle possible with a PRP but could not be possible with a truly random bijection.
\end{note}

Most ciphers are designed under the assumption of black-box attack context thus their implementation must be extremely careful. The attacker could for example measure the time of encryption and exploit this information which was not considered during the design of the cipher.

\subsubsection{Gray-Box}

Gray-box attack context is closer to the reality since we do not encrypt using oracles but rather using some physical devices which run a specific implementation of given cipher. In the gray box model, cipher is thus supposed to be implemented and run while the attacker can observe some information related to cipher's byproducts as well.

Such information leaks can be modelled as a set of distributions conditioned by byproducts and are referred to as {\em leakage model}. There is an area of theoretical cryptography studying leakages -- {\em Leakage Resilient Cryptography}. A practical attack exploiting such leaks will be described in Section \ref{sec:side}.   %?% LRC introduced by?

\subsubsection{White-Box}

Finally there is the most extreme case where the attacker has full control over the execution environment -- the white-box attack context. Full control means that the attacker can step the algorithm and directly observe and alter byproducts in the memory or even skip, alter and insert instructions. Clearly, the key cannot be kept in its original form, it must be somehow hidden but still usable for encryption.
