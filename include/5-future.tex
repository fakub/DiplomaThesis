\chapter{Future Work}
\label{chap:future}

\section{Another Interesting AES White-Box Implementation}
\label{sec:magicimpl}

Bos et al.\ \cite{bos2015differential} mention another interesting AES white-box implementation in their paper. This particular implementation by Eloi Vanderbéken was first published only as a Windows binary executable challenge at NoSuchCon 2013 conference\footnote{\link{http://www.nosuchcon.org/2013/}}, where nobody was able to recover the key. Note that this implementation uses an externally-encoded variant of AES (as outlined in Section \ref{sec:constrwbaes}, in particular in Equation \ref{eq:extenc}), hence it is {\em not compliant with AES anymore}. It follows that any SCA-like attack is impossible, since the internal AES handles encoded inputs, where the encoding is unknown.

Soon after the conference, the author released source codes\footnote{\link{http://pastebin.com/MvXpGZts}}, which allows one to avoid reverse engineering of the binary. Moreover, it allows one to modify the code to run natively on Linux. Philippe Teuwen, a co-author of Bos et al.\ \cite{bos2015differential}, studied this implementation deeper and finally was able to recover the key \cite{teuwen2015nscwriteups}.

In his write-up, Teuwen also describes the structure, which appears to avoid usage of $4$-bit random bijections (in WBAES by Chow et al.\ \cite{chow2002aes}) and uses $8$-bit ones instead. Note that even without any non-linearity check, it is far less likely that any bit of output of a random $8$-bit bijection would provide a reasonable level of linearity compared to a random $4$-bit bijection, as outlined in Section \ref{sec:attempt}.

However, we did not either modify the source code to make it AES-compliant, nor did run the attack, hence it remains as a possible topic for further research.

%?%
% \section{What next?}