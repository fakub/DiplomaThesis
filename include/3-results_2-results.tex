\section{Results}
\label{sec:results}

First of all we used our toolkit to attack our straightforward C++ implementation of AES, here named {\tt naiveAES}, which served just as a proof-of-concept. Then we performed the most interesting attack which Bos et al.\ \cite{bos2015differential} did -- we successfuly attacked Klinec's C++ implementation \cite{klinec2013implementation} of WBAES by Chow et al.\ \cite{chow2003aes}, here named {\tt KlinecWBAES}.

Note that the following section giving results of attack against {\tt naiveAES} does not fully cover many details, these are given separately in the subsequent section with results of the attack against {\tt KlinecWBAES}.


% ======================================================================
% ===   n a i v e A E S                                              ===
% ======================================================================

\subsection{\tt naiveAES}
\label{sec:naiveaes}

In order to prove the concept of both SCA algorithms presented in this thesis, we first attacked our straightforward AES implementation {\tt naiveAES} with both of them.

\subsubsection{CPA Attack against {\tt naiveAES}}
	
	What experiment we made \ldots
	% We repeated CPA attack (i.e. Algorithm \ref{alg:cpa}) with $10$ different keys and for different amounts of plaintexts. See results in Table \ref{tab:naiveaes}.
	%~ CPA attack required $x$ traces, bitwise DPA attack required $y$ traces.
	
	\begin{table}[H]
		\begin{center}
		\begin{tabular}{| c | c | c | c |}
			\hline
			Results & of attack & against {\tt naiveAES} & . \\
			\hline
		\end{tabular}
		\end{center}
	\caption{CPA attack against {\tt naiveAES}}
	\label{tab:naiveaescpa}
	\end{table}
	

\subsubsection{Bit-Wise DPA Attack against {\tt naiveAES}.}
	
	Algorithm \ref{alg:bitwisedpa}.
	
	\begin{table}[H]
		\begin{center}
		\begin{tabular}{| c | c | c | c |}
			\hline
			Results & of attack & against {\tt naiveAES} & . \\
			\hline
		\end{tabular}
		\end{center}
	\caption{Bit-Wise DPA attack against {\tt naiveAES}.}
	\label{tab:naiveaesdca}
	\end{table}
	%?% vyzkoušet rijinv na naiveAES, nemělo by jít, a taky nejde


% ======================================================================
% ===   K l i n e c W B A E S                                        ===
% ======================================================================

\subsection{\tt KlinecWBAES}
\label{sec:klinecwbaes}
	
	CPA attack did not work for {\tt KlinecWBAES} \ldots
	
	We attacked {\tt KlinecWBAES} with bit-wise DPA attack using $1024$ traces, see results in Table \ref{tab:klinecsbox}.
	In this table we give percentual gap of the best candidate and rank of the true key byte for each key byte and target bit.
	
	\begin{note}
	\label{note:tailrank}
		There are remarkably many tail ranks (e.g.\ between $250$ and $255$) in results in Table \ref{tab:klinecsbox}.
	\end{note}
	
	\begin{landscape}
	\begin{table}[H]   % přidat počty úspěšnejch?
		\begin{center}
		\begin{tabular}{| c | r@{.} l@{\quad}r | r@{.} l@{\quad}r | r@{.} l@{\quad}r | r@{.} l@{\quad}r | r@{.} l@{\quad}r | r@{.} l@{\quad}r | r@{.} l@{\quad}r | r@{.} l@{\quad}r |}
	\hline
	\multirow{2}{*}{Byte} & \multicolumn{24}{c|}{Target bits (percentual gap and rank)} \\
	\cline{2-25}
	~ & \multicolumn{3}{c|}{1. bit} & \multicolumn{3}{c|}{2. bit} & \multicolumn{3}{c|}{3. bit} & \multicolumn{3}{c|}{4. bit} & \multicolumn{3}{c|}{5. bit} & \multicolumn{3}{c|}{6. bit} & \multicolumn{3}{c|}{7. bit} & \multicolumn{3}{c|}{8. bit} \\
	\hline
	\hline
	1.&33&4&{$\blacksquare$}&4&0&55&3&1&90&53&2&{$\blacksquare$}&5&0&149&4&5&207&4&2&224&20&4&{$\blacksquare$}\\
	\hline
	2.&0&2&248&11&6&218&2&8&239&0&7&244&11&5&247&45&6&{$\blacksquare$}&2&9&251&10&8&247\\
	\hline
	3.&40&7&{$\blacksquare$}&15&1&212&30&8&{$\blacksquare$}&10&5&25&2&8&230&47&8&{$\blacksquare$}&8&9&99&17&2&{$\blacksquare$}\\
	\hline
	4.&45&2&{$\blacksquare$}&9&0&252&1&7&226&4&1&247&19&4&{$\blacksquare$}&3&8&255&5&1&241&4&2&252\\
	\hline
	5.&4&4&247&1&3&104&45&4&{$\blacksquare$}&4&0&225&2&7&229&14&7&{$\blacksquare$}&1&2&225&0&5&249\\
	\hline
	6.&6&5&252&4&9&255&54&8&{$\blacksquare$}&2&9&241&12&6&242&48&7&{$\blacksquare$}&4&7&4&13&9&255\\
	\hline
	7.&11&5&47&3&7&233&16&0&{$\blacksquare$}&0&6&228&38&1&{$\blacksquare$}&8&8&{\weak$\blacksquare$}&37&7&{$\blacksquare$}&44&0&{$\blacksquare$}\\
	\hline
	8.&54&1&{$\blacksquare$}&0&9&253&4&5&253&49&5&{$\blacksquare$}&5&3&255&2&0&251&1&3&1&50&1&{$\blacksquare$}\\
	\hline
	9.&4&7&224&3&3&196&22&1&231&0&7&249&3&1&253&1&5&238&4&7&{\weak$\blacksquare$}&22&4&253\\
	\hline
	10.&43&7&{$\blacksquare$}&68&6&{$\blacksquare$}&0&0&255&2&8&245&4&1&255&6&3&{\weak$\blacksquare$}&5&9&234&52&6&{$\blacksquare$}\\
	\hline
	11.&6&6&245&5&7&{\weak$\blacksquare$}&4&0&250&33&3&{$\blacksquare$}&5&6&190&4&4&255&8&3&236&15&8&{$\blacksquare$}\\
	\hline
	12.&8&7&254&4&7&255&51&8&{$\blacksquare$}&1&0&255&47&4&{$\blacksquare$}&50&7&{$\blacksquare$}&45&6&{$\blacksquare$}&14&0&{$\blacksquare$}\\
	\hline
	13.&0&3&241&1&6&{\weak$\blacksquare$}&14&6&254&4&3&190&7&1&160&6&9&193&28&4&{$\blacksquare$}&40&9&{$\blacksquare$}\\
	\hline
	14.&4&4&235&41&3&{$\blacksquare$}&0&0&254&47&1&{$\blacksquare$}&0&2&255&0&7&2&15&2&{$\blacksquare$}&5&9&255\\
	\hline
	15.&27&7&{$\blacksquare$}&58&5&{$\blacksquare$}&5&9&246&1&1&195&1&0&255&27&8&{$\blacksquare$}&2&7&246&14&7&155\\
	\hline
	16.&1&3&252&6&0&255&0&2&254&5&8&{\weak$\blacksquare$}&1&9&251&7&6&245&2&3&235&52&9&{$\blacksquare$}\\
	\hline
\end{tabular}
		\end{center}
	\caption{Bit-Wise DPA attack against {\tt KlinecWBAES} using $1024$ traces. Percentual gap of the best candidate and rank of the true key byte is given for each key byte and target bit. Rank of the true candidate goes from $0$ while $0$ (i.e.\ top position) is replaced with $\blacksquare$ in order to be emphasized.}
	\label{tab:klinecsbox}
	\end{table}
	\end{landscape}
	
%~ kolik traců, jak dlouho to běželo, že jsem zjistil adresu kde to leakuje
%~ co jsem použil a co vypadlo, de facto takovej protokol
%~ White-Box-AES implementation was only broken by bitwise DPA attack and many interesting things occured.
% provést útok s několikrát nagenerovanejma tabulkama (v manuálu pak popsat co je pro to potřeba udělat)

%!% další implementace??
%~ proč to s tou javou nešlo ... (ale vyzkoušet)
