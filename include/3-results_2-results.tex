\section{Results}
\label{sec:results}

First of all, we used our toolkit to attack our straightforward C++ implementation of AES, here named {\tt naiveAES}, which served just as a proof-of-concept, see results in Section \ref{sec:naiveaes}. Note that the section does not fully cover many details, these are given separately later.

Then we performed the most interesting attack which Bos et al.\ \cite{bos2015differential} did -- we successfuly attacked Klinec's C++ implementation \cite{klinec2013implementation} of WBAES by Chow et al.\ \cite{chow2002aes}, here named {\tt KlinecWBAES}. See results and comments in Section \ref{sec:klinecwbaes}.


% ==============================================================================
% ===   n a i v e A E S                                                      ===
% ==============================================================================

\subsection{\tt naiveAES}
\label{sec:naiveaes}

In order to prove the concept of both SCA algorithms presented in this thesis, we first attacked our straightforward AES implementation {\tt naiveAES} with both of them. Both attacks are expected to work since {\tt naiveAES} stores all intermediate results needed for each attack.

\subsubsection{CPA Attack against {\tt naiveAES}}
	
	What experiment we made \ldots
	% We repeated CPA attack (i.e. Algorithm \ref{alg:cpa}) with $10$ different keys and for different amounts of plaintexts. See results in Table \ref{tab:naiveaes}.
	%~ CPA attack required $x$ traces, bitwise DPA attack required $y$ traces.
	
	\begin{table}[H]
		\begin{center}
		\begin{tabular}{| c | c | c | c |}
			\hline
			Results & of attack & against {\tt naiveAES} & . \\
			\hline
		\end{tabular}
		\end{center}
	\caption{CPA attack against {\tt naiveAES}}
	\label{tab:naiveaescpa}
	\end{table}
	

\subsubsection{Bit-Wise DPA Attack against {\tt naiveAES}}
	
	Algorithm \ref{alg:bitwisedpa} \ldots
	
	\begin{table}[H]
		\begin{center}
		\begin{tabular}{| c | c | c | c |}
			\hline
			Results & of attack & against {\tt naiveAES} & . \\
			\hline
		\end{tabular}
		\end{center}
	\caption{Bit-Wise DPA attack against {\tt naiveAES}.}
	\label{tab:naiveaesdca}
	\end{table}
	%?% vyzkoušet rijinv na naiveAES, nemělo by jít, a taky nejde


% ==============================================================================
% ===   K l i n e c W B A E S                                                ===
% ==============================================================================

\subsection{\tt KlinecWBAES}
\label{sec:klinecwbaes}

First we tried to attack {\tt KlinecWBAES} with CPA attack, too, but it did not break any key byte at all. Therefore the rest of our results focuses exclusively on bit-wise DPA attack.

\subsubsection{Bit-Wise DPA Attack against {\tt KlinecWBAES}}
	
	We used $1024$ traces for the attack, see results in Table \ref{tab:klinecsbox}. In this table we give percentual gap of the best candidate to the second candidate together with rank of the true key byte, for each key byte and target bit.
	
	\afterpage{
		\clearpage   % To flush out all floats, might not be what you want
		\begin{landscape}
		\begin{table}[H]
			\begin{center}
			\begin{tabular}{| c | r@{.} l@{\quad}r | r@{.} l@{\quad}r | r@{.} l@{\quad}r | r@{.} l@{\quad}r | r@{.} l@{\quad}r | r@{.} l@{\quad}r | r@{.} l@{\quad}r | r@{.} l@{\quad}r |}
	\hline
	\multirow{2}{*}{Byte} & \multicolumn{24}{c|}{Target bits (percentual gap\quad rank)} \\
	\cline{2-25}
	~ & \multicolumn{3}{c|}{1. bit} & \multicolumn{3}{c|}{2. bit} & \multicolumn{3}{c|}{3. bit} & \multicolumn{3}{c|}{4. bit} & \multicolumn{3}{c|}{5. bit} & \multicolumn{3}{c|}{6. bit} & \multicolumn{3}{c|}{7. bit} & \multicolumn{3}{c|}{8. bit} \\
	\hline
	1.&33&4&$\blacksquare$&4&0&55&3&1&90&53&2&$\blacksquare$&5&0&149&4&5&207&4&2&224&20&4&$\blacksquare$\\
	\hline
	2.&0&2&248&11&6&218&2&8&239&0&7&244&11&5&247&45&6&$\blacksquare$&2&9&251&10&8&247\\
	\hline
	3.&40&7&$\blacksquare$&15&1&212&30&8&$\blacksquare$&10&5&25&2&8&230&47&8&$\blacksquare$&8&9&99&17&2&$\blacksquare$\\
	\hline
	4.&45&2&$\blacksquare$&9&0&252&1&7&226&4&1&247&19&4&$\blacksquare$&3&8&255&5&1&241&4&2&252\\
	\hline
	5.&4&4&247&1&3&104&45&4&$\blacksquare$&4&0&225&2&7&229&14&7&$\blacksquare$&1&2&225&0&5&249\\
	\hline
	6.&6&5&252&4&9&255&54&8&$\blacksquare$&2&9&241&12&6&242&48&7&$\blacksquare$&4&7&4&13&9&255\\
	\hline
	7.&11&5&47&3&7&233&16&0&$\blacksquare$&0&6&228&38&1&$\blacksquare$&8&8&$\blacksquare$&37&7&$\blacksquare$&44&0&$\blacksquare$\\
	\hline
	8.&54&1&$\blacksquare$&0&9&253&4&5&253&49&5&$\blacksquare$&5&3&255&2&0&251&1&3&1&50&1&$\blacksquare$\\
	\hline
	9.&4&7&224&3&3&196&22&1&231&0&7&249&3&1&253&1&5&238&4&7&$\blacksquare$&22&4&253\\
	\hline
	10.&43&7&$\blacksquare$&68&6&$\blacksquare$&0&0&255&2&8&245&4&1&255&6&3&$\blacksquare$&5&9&234&52&6&$\blacksquare$\\
	\hline
	11.&6&6&245&5&7&$\blacksquare$&4&0&250&33&3&$\blacksquare$&5&6&190&4&4&255&8&3&236&15&8&$\blacksquare$\\
	\hline
	12.&8&7&254&4&7&255&51&8&$\blacksquare$&1&0&255&47&4&$\blacksquare$&50&7&$\blacksquare$&45&6&$\blacksquare$&14&0&$\blacksquare$\\
	\hline
	13.&0&3&241&1&6&$\blacksquare$&14&6&254&4&3&190&7&1&160&6&9&193&28&4&$\blacksquare$&40&9&$\blacksquare$\\
	\hline
	14.&4&4&235&41&3&$\blacksquare$&0&0&254&47&1&$\blacksquare$&0&2&255&0&7&2&15&2&$\blacksquare$&5&9&255\\
	\hline
	15.&27&7&$\blacksquare$&58&5&$\blacksquare$&5&9&246&1&1&195&1&0&255&27&8&$\blacksquare$&2&7&246&14&7&155\\
	\hline
	16.&1&3&252&6&0&255&0&2&254&5&8&$\blacksquare$&1&9&251&7&6&245&2&3&235&52&9&$\blacksquare$\\
	\hline
\end{tabular}   %!% obrátit pořadí sloupečků !!!
			\end{center}
		\caption{Bit-Wise DPA attack against {\tt KlinecWBAES} using $1024$ traces. Percentual gap of the best candidate and rank of the true key byte is given for each key byte and target bit. Rank of the true candidate goes from $0$ while $0$ (i.e.\ the top position) is replaced with $\blacksquare$ or {\weak$\blacksquare$} in order to be well emphasized, for strong or weak candidate, respectively.}
		\label{tab:klinecsbox}
		\end{table}
		\end{landscape}
	}
	
	On average, there are about $31\%$ of successful strong candidates and about $36\%$ of both weak and strong ones (see Note \ref{note:strong} for strong vs.\ weak candidates).
	
	Compared to the results of Bos et al.\ \cite{bos2015differential}, we successfuly reproduced one half of their attack. The second half is given in Section \ref{sec:rijinv} because it requires some improvements to be introduced.
	
	\begin{note}
	\label{note:tailrank}
		As already noted by the original authors, we can see remarkably many tail ranks (e.g.\ between $250$ and $255$) in results in Table \ref{tab:klinecsbox}.
	\end{note}

% jak dlouho to běželo, že jsem zjistil adresu kde to leakuje


% ==============================================================================
% ===   B a c i n s k a   W B A E S +                                        ===
% ==============================================================================

% proč to s tou javou nešlo:
% neimplementuje AES ale "zlepšenou" verzi
% key-dependent SBoxy => útok tak jak je ani fungovat nemůže
