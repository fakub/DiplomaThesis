\section{Results}
\label{sec:results}

First we used our toolkit to attack our straightforward implementation of AES, here named {\tt naiveAES}, which served just as a proof-of-concept. Then we performed the most interesting attack which Bos et al.\ \cite{bos2015differential} did -- we successfuly attacked Klinec's implementation \cite{klinec2013implementation} of WBAES by Chow et al.\ \cite{chow2003aes}, here named {\tt KlinecWBAES}.

Note that the following Section \ref{sec:naiveaes} giving results of attack against {\tt naiveAES} does not fully cover many details, these are given separately in the subsequent Section \ref{sec:klinecwbaes} with results of the attack against {\tt KlinecWBAES}.


% ======================================================================
% ===   n a i v e A E S                                              ===
% ======================================================================

\subsection{\tt naiveAES}
\label{sec:naiveaes}

In order to prove the concept of both SCA algorithms presented in this thesis, we first attacked our straightforward AES implementation {\tt naiveAES} with both of them.

\subsubsection{CPA against {\tt naiveAES}}
	
	We repeated CPA attack (i.e. Algorithm \ref{alg:cpa}) with $10$ different keys and for different amounts of plaintexts. See results in Table \ref{tab:naiveaes}.
	
	\begin{table}[H]
		\begin{center}
		\begin{tabular}{| c | c | c | c |}
			\hline
			~ & \multicolumn{2}{c|}{\bf $n\times n$ squares} & {\bf Computational} \\
			~ & \multicolumn{1}{c}{Lower bound} & \multicolumn{1}{c|}{Upper bound} & {\bf Power}\\
			\hline
			2D & \multicolumn{2}{c|}{See see} & Universal \\
			\hline
			2D & see & see & Unknown \\
			\hline
		\end{tabular}
		\end{center}
	\caption{CPA against }
	\label{tab:naiveaes}
	\end{table}

\subsubsection{Bit-Wise DPA against {\tt naiveAES}}
	
	Algorithm \ref{alg:bitwisedpa}
	
	%?% vyzkoušet rijinv na naiveAES, nemělo by jít


% ======================================================================
% ===   K l i n e c W B A E S                                        ===
% ======================================================================

\subsection{\tt KlinecWBAES}
\label{sec:klinecwbaes}



%~ co jsem použil za implmentace, kolik traců, jak dlouho to běželo, že jsem zjistil adresu kde to leakuje
%~ teoreticky by mělo bejt popsáno v předch. kapitole, tady už bych jenom zmínil co jsem použil a co vypadlo
%~ a proč to s tou javou nešlo ... (ale vyzkoušet)
%~ de facto takovej protokol
%~ Two (?) implementations will be considered for the attack:
%~ \begin{itemize}
	%~ \item our naive AES implementation (C++), and
	%~ \item Klinec's implementation of White-Box AES \cite{klinec2013white} (C++).
%~ \end{itemize}
%~ NaiveAES implementation was broken by both algorithms, CPA attack required $x$ traces, bitwise DPA attack required $y$ traces.
%~ White-Box-AES implementation was only broken by bitwise DPA attack and many interesting things occured.
% provést útok s několikrát nagenerovanejma tabulkama (v manuálu pak popsat co je pro to potřeba udělat)

