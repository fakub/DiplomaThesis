\chapter{Practical Attack \& Results}
\label{chap:results}

%~ nemá bejt manuál pro moje scripty, ale
%~ co jsem použil za implmentace, kolik traců, jak dlouho to běželo, že jsem zjistil adresu kde to leakuje
%~ že jsem se pokoušel neúspěšně útočit na netriv. indexy při čtení z tabulek
%~ a proč to s tou javou nešlo ... (ale vyzkoušet)
%~ de facto takovej protokol
%~ pokud bych našel v kódu kde to leakuje, uměl to vyčíst z indexů popř. nák přímo z tabulek -> 4 Analysis

%~ Two (?) implementations will be considered for the attack:
%~ \begin{itemize}
	%~ \item our naive AES implementation (C++), and
	%~ \item Klinec's implementation of White-Box AES \cite{klinec2013white} (C++).
%~ \end{itemize}
%~ The attack will perform as follows:
%~ \begin{enumerate}
	%~ \item acquire memory traces,
	%~ \item filter memory traces,
	%~ \item run DCA attack.
%~ \end{enumerate}
%~ In the following sections (?) we describe these steps in detail, finally we present practical results of the attack.

%~ teoreticky by mělo bejt popsáno v předch. kapitole, tady už bych jenom zmínil co jsem použil a co vypadlo
