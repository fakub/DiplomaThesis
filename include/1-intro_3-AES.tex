\section{Advanced Encryption Standard}
\label{sec:aes}

In this section, we present a currently standardized block cipher. Let us begin with how it originated. In 1997, Curtin and Dolske were the first to publicly announce \cite{curtin1998brute} that they have cracked Data Encryption Standard (DES, \cite{fips1977des}) -- a symmetric block cipher which had been in use for $20$ years. In the same year, the National Institute of Standards and Technology (NIST) initiated a selection process for a new encryption standard. During the process, DES was partially replaced with Triple-DES described in the ANSI draft X9.52 \cite{americantripple}.

In 1998, NIST chose $15$ candidates and asked cryptology community for help with analyzing the candidates. After a careful analysis, NIST proposed Rijndael\footnote{From authors' names: Vincent Rijmen and Joan Daemen.} \cite{daemen1999aes} as the new standard called {\em Advanced Encryption Standard} (AES) \cite{fips2001aes} on November 26, 2001. The standard specifies the Rijndael algorithm with block-length of $128$ bits and key-lengths of $128$, $192$ and $256$ bits. Note that the original Rijndael algorithm \cite{daemen2013rijndael} describes more block- and key-length variants. On the other hand, we will only consider its $128$-$128$-bit variant in this thesis, unless stated otherwise.

Today, almost 15 years later, AES is used worldwide and, despite extensive analysis, it is still beleived to have considerable security margin.


% ==============================================================================
% ===   A E S   B U I L D I N G   B L O C K S                                ===
% ==============================================================================

\subsection{AES Building Blocks}

AES encryption\footnote{Decryption will be treated separately at the end of this section.} consists of $10$ rounds, each round (except for the last one) consists of $4$ different operations which will be described later. All of these operations work with $128$-bit words, let us call them {\em states}. Before we start, let us present a couple of different state representations. 

First, we split its $128$ bits into $16$ bytes denoted by $A_{0,0}$, $A_{1,0}$, $A_{2,0}$, $A_{3,0}$, $A_{0,1}$, $\ldots$, $A_{3,1}$, $\ldots$, $A_{3,3}$, respectively, obtaining a $4\times 4$ array, where $A_{i,j}$ stands for an element in $i$-th row and $j$-th column.

Let us further denote $F = \GF(2)[x]/x^8+x^4+x^3+x+1$, which is a field of polynomials with integral coefficients reduced modulo $2$ (i.e., $0$, $1$), these polynomials are further reduced modulo $x^8+x^4+x^3+x+1$. This field has $2^8$ elements and is usually referred to as {\em Rijndael's field}. In the context of AES, it might be denoted also by $\GF(2^8)$.

\nomenclature{$\GF(p^n)$}{Galois Field with $p^n$ elements, $p$ prime}
\nomenclature{$F[x]$}{Ring of polynomials over $F$}

Each byte $A = a_7a_6\ldots a_0$, $a_i\in\GF(2)$, is then considered as an element of $F$ simply as $a_7x^7 + a_6x^6 + a_5x^5 + a_4x^4 + a_3x^3 + a_2x^2 + a_1x + a_0$.  Note that byte may also be written in hexadecimal notation, and still considered as an element of $F$. See Figure~\ref{fig:aesnot} with a few of possible AES state representations.

\begin{figure}[H]
\[
\arraycolsep=.5em\def\arraystretch{1.5}
	01010001\,11010110\,\ldots\;\sim\;\texttt{51}\,\texttt{d6}\,\ldots\;\sim\;
	\begin{array}{|c|c|c|c|}
		\hline
		x^6 + x^4 + 1 & \ldots & \ldots & \ldots \\
		\hline
		x^7 + x^6 + x^4 + x^2 + x & \ldots & \ldots & \ldots \\
		\hline
		\ldots & \ldots & \ldots & \ldots \\
		\hline
		\ldots & \ldots & \ldots & \ldots \\
		\hline
	\end{array}
\]
\caption{A few of possible AES state representations.}
\label{fig:aesnot}
\end{figure}

\noindent
Now let us move on to definitions of the $4$ operations.

\subsubsection{SubBytes}
	
	$\SubBytes$ operation, often referred to as {\em SBox}, will be denoted by $S$. It is defined byte-wise, i.e., $S: F \rightarrow F$. For $A\in F$, let $A'$ stand for
	\begin{itemize}
		\item $A^{-1}$, if $A \neq 0$,
		\item $0$, if $A = 0$.
	\end{itemize}
	
	\begin{note}
	\label{note:rijinv}
		Even though $A'$ is not a proper field inverse, it is often referred to as {Rijndael inverse} of $A$. Note that Rijndael inverse (taken as mapping) is invertible.
	\end{note}
	
	Before we define $\SubBytes$, we have to emphasize that the following multiplication is {\em not} performed in $F$, but in the ring of binary polynomials $\GF(2)[x]$ instead. $\SubBytes$ is defined as follows:
	\newcommand{\defsbox}{S(A) = (x^4 + x^3 + x^2 + x + 1) \cdot A' + x^6 + x^5 + x + 1 \pmod{x^8+1}}
	\begin{equation}
	\label{eq:sbox}
		\defsbox .
	\end{equation}
	
	\begin{note}
	\label{note:sboxinv}
		Although $x^8+1$ is not irreducible (it holds $x^8+1 = (x+1)^8$), $x^4 + x^3 + x^2 + x + 1$ is coprime with $x^8+1$, therefore it has an inverse modulo $x^8+1$. It follows that $\SubBytes$ is an invertible bijection.
	\end{note}
	
	$\SubBytes$ is the only confusion element in AES -- indeed, all the other operations will be linear (i.e., from a limited subset of mappings; see Principle~\ref{pri:shannon}). Since $\SubBytes$ is the only source of nonlinearity in AES, it shall have excellent nonlinearity properties. This was the major weakness of DES, its predecessor. A practical linear attack against DES was introduced by Matsui \cite{matsui1993linear} in 1993 and analyzed in detail and successfuly performed by Junod in 2001 \cite{junod2001complexity}.
	
	\begin{remark}
	\label{rem:sboxtable}
		Since $\SubBytes$ only has $256$ possible inputs, it is usually precomputed and implemented as a lookup table, obviously for performance reasons.
	\end{remark}
	
	The reason why $\SubBytes$ was not introduced the other way around (i.e., as a lookup table with good nonlinearity properties) is the simple algebraic form of $S$, which enables mathematical analysis as well as statistical.
	
	\begin{remark}
	\label{rem:sboxaff}
		$\SubBytes$ is actually an affine mapping (in the vector space $F$ over $\GF(2)$) of Rijndael inverse of its input.
	\end{remark}

\subsubsection{ShiftRows}
	
	$\ShiftRows$ operation is very simple -- it only cyclically shifts rows of the state array. It moves $i$-th row by $i$ positions to the left where rows are numbered from $0$; see Figure~\ref{fig:shiftrows}. $\ShiftRows$ is obviously invertible.
	
	\begin{figure}[H]
	\[
	\arraycolsep=.5em\def\arraystretch{1.5}
		\begin{array}{|c|c|c|c|}
			\hline
			A_{0,0} & \cg{.92} A_{0,1} & \cg{.84} A_{0,2} & \cg{.76} A_{0,3} \\
			\hline
			A_{1,0} & \cg{.92} A_{1,1} & \cg{.84} A_{1,2} & \cg{.76} A_{1,3} \\
			\hline
			A_{2,0} & \cg{.92} A_{2,1} & \cg{.84} A_{2,2} & \cg{.76} A_{2,3} \\
			\hline
			A_{3,0} & \cg{.92} A_{3,1} & \cg{.84} A_{3,2} & \cg{.76} A_{3,3} \\
			\hline
		\end{array}
		\quad\xrightarrow{\ShiftRows}\quad
		\begin{array}{|c|c|c|c|}
			\hline
			A_{0,0} & \cg{.92} A_{0,1} & \cg{.84} A_{0,2} & \cg{.76} A_{0,3} \\
			\hline
			\cg{.92} A_{1,1} & \cg{.84} A_{1,2} & \cg{.76} A_{1,3} & A_{1,0} \\
			\hline
			\cg{.84} A_{2,2} & \cg{.76} A_{2,3} & A_{2,0} & \cg{.92} A_{2,1} \\
			\hline
			\cg{.76} A_{3,3} & A_{3,0} & \cg{.92} A_{3,1} & \cg{.84} A_{3,2} \\
			\hline
		\end{array}
	\]
	\caption{$\ShiftRows$ operation on a state array.}
	\label{fig:shiftrows}
	\end{figure}
	
	The design motivation for $\ShiftRows$ was to introduce inter-column diffusion. Note that $\ShiftRows$ does not diffuse bytes internally -- the following operation is here for this purpose.

\subsubsection{MixColumns}
	
	$\MixColumns$ operates on the state array in a column-wise manner. Let us pick a column from the state array and denote its elements by $A_0, A_1, A_2, A_3$, respectively. $\MixColumns$ considers it as a polynomial $a(z) = A_0 + A_1 z + A_2 z^2 + A_3 z^3$ -- an element of $F[z]$. Let $b(z)$ denote the output of $\MixColumns$, then it is defined as
	\begin{equation}
	\label{eq:mixcolpoly}
		b(z) = c(z) \cdot a(z) \pmod{z^4+1}
	\end{equation}
	where $c(z) = \texttt{03}\cdot z^3 + \texttt{01}\cdot z^2 + \texttt{01}\cdot z^2 + \texttt{02}$. Note that here we used hexadecimal notation for elements of $F$, which can also be viewed as polynomials. To avoid confusion, we used $z$ instead of $x$.
	
	Note that $c(z)$ is coprime with $z^4+1$, therefore invertible modulo $z^4+1$. It follows that $\MixColumns$ is invertible. The goal of $\MixColumns$ is to introduce inter-byte diffusion within a column.
	
	Later we will make use of its different, linear algebra representation. Let us denote $b(z) = B_0 + B_1 z + B_2 z^2 + B_3 z^3$. Modular multiplication in Equation~\ref{eq:mixcolpoly}, which is performed in $F[z]$, can be rewritten into a matrix multiplication back in $F$, indeed
	\begin{equation}
	\label{eq:mixcolmatr}
		\begin{pmatrix}
			B_0 \\ B_1 \\ B_2 \\ B_3
		\end{pmatrix}
		=
		\begin{pmatrix}
			\texttt{02} & \texttt{03} & \texttt{01} & \texttt{01} \\
			\texttt{01} & \texttt{02} & \texttt{03} & \texttt{01} \\
			\texttt{01} & \texttt{01} & \texttt{02} & \texttt{03} \\
			\texttt{03} & \texttt{01} & \texttt{01} & \texttt{02}
		\end{pmatrix}
		\begin{pmatrix}
			A_0 \\ A_1 \\ A_2 \\ A_3
		\end{pmatrix}.
	\end{equation}

\subsubsection{AddRoundKey}
	
	Given a $128$-bit round key, $\AddRoundKey$ operation simply bitwise XORs it on the state. The round key is derived from the encryption key using $\KeySchedule$ routine which follows.
	
	$\AddRoundKey$ is the only step which employs keying material. Note that it is its own inverse.

\subsubsection{KeySchedule}
	
	$\KeySchedule$ is a routine which expands $128$-bit encryption key into $11$-times longer $\ExpandedKey$. $128$-bit blocks of $\ExpandedKey$ serve as round keys. Their position within $\ExpandedKey$ is indexed from $0$.
	
	\begin{note}
	\label{note:expkey}
		The most important consequence for us is that $\ExpandedKey$ begins with plain encryption key.
	\end{note}
	
	We omit further details here, because $\KeySchedule$ is specified rather in an algorithmic than algebraic manner; see \cite[pp.43-45]{daemen2013rijndael} for detail.


% ==============================================================================
% ===   A E S   A L G O R I T H M   D E S C R I P T I O N                    ===
% ==============================================================================

\subsection{AES Algorithm Description}

AES, as already stated, consists of $10$ rounds of $4$ operations except for the last one where $\MixColumns$ is missing; see the following algorithm.

\begin{alg}
\label{alg:aes}
	Given a $128$-bit plaintext and key, return its AES encryption.
	\begin{algorithmic}[1]
		\Function{AES\_Encryption}{$Plaintext,Key$}
			\State $\ExpandedKey \gets \KeySchedule(Key)$
			\State $State \gets Plaintext$ \label{line:stateplain}
			\State $\AddRoundKey(State, \ExpandedKey[0])$ \label{line:addrk}
			\For{$Round = 1 \to 9$} \label{line:aesfor}
				\State $\SubBytes(State)$
				\State $\ShiftRows(State)$
				\State $\MixColumns(State)$
				\State $\AddRoundKey(State, \ExpandedKey[Round])$
			\EndFor
			\State $\SubBytes(State)$
			\State $\ShiftRows(State)$
			\State $\AddRoundKey(State, \ExpandedKey[10])$
			\State\Return $State$
		\EndFunction
	\end{algorithmic}
\end{alg}

\begin{note}
	Since we have commented on every step about how it can be inverted, AES encryption algorithm is hence also invertible. We get AES decryption algorithm by executing inverse counterparts of AES encryption steps in reverse order.
\end{note}


% ==============================================================================
% ===   A E S   I M P L E M E N T A T I O N   N O T E                        ===
% ==============================================================================

\subsection{AES Implementation Note}
\label{sec:aeslookup}

As stated in Remark~\ref{rem:sboxtable}, $\ShiftRows$ can be implemented as a lookup table for performance reasons. Unlike $\SubBytes$, $\MixColumns$ operates on a substantially larger domain, therefore it cannot be directly implemented as a lookup table. But we can utilize its matrix representation, see Equation~\ref{eq:mixcolmatr}. Let us denote
\[
	\C =
	\begin{pmatrix}
		\texttt{02} & \texttt{03} & \texttt{01} & \texttt{01} \\
		\texttt{01} & \texttt{02} & \texttt{03} & \texttt{01} \\
		\texttt{01} & \texttt{01} & \texttt{02} & \texttt{03} \\
		\texttt{03} & \texttt{01} & \texttt{01} & \texttt{02}
	\end{pmatrix} ,
\]
we can rewrite the equation to the form
\begin{align*}
	\begin{pmatrix}
			B_0 \\ B_1 \\ B_2 \\ B_3
		\end{pmatrix}
		&=
		\C
		\begin{pmatrix}
			A_0 \\ 0 \\ 0 \\ 0
		\end{pmatrix}
		+
		\C
		\begin{pmatrix}
			0 \\ A_1 \\ 0 \\ 0
		\end{pmatrix}
		+
		\C
		\begin{pmatrix}
			0 \\ 0 \\ A_2 \\ 0
		\end{pmatrix}
		+
		\C
		\begin{pmatrix}
			0 \\ 0 \\ 0 \\ A_3
		\end{pmatrix}
		= \\
		&=
		\C_0 \cdot A_0   
		 +
		\C_1 \cdot A_1  
		 +
		\C_2 \cdot A_2 
		 +
		\C_3 \cdot A_3 ,
\end{align*}
where $\C_i$ denotes $i$-th column of $\C$.

Note that each addend only depends on single byte, i.e., there are just $256$ possible input values for each. Therefore $\MixColumns$ can be implemented as a fourtuple of lookup tables, the result is then simply a bitwise XOR of their $4$-byte output values.

This was just to give an idea about how AES operations can be turned into lookup tables. These can be further improved, see \cite[Chapter~4]{daemen2013rijndael} for detail.
